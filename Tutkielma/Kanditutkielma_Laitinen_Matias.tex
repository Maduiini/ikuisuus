% !TeX encoding = utf8
%
% [ Tiedostossa käytetty merkistö on utf8, vaihtoehtoisesti voisi olla esim.]
% [ ISO 8859-1 eli Latin 1. Ylläoleva rivi ]
% [ tarvitaan, jos käyttää MiKTeX-paketin mukana tulevaa TeXworks-editoria. ]
%
% TIETOTEKNIIKAN KANDIDAATINTUTKIELMA
%
% Yksinkertainen LaTeX2e-mallipohja kandidaatintutkielmalle.
% Käyttää Antti-Juhani Kaijanahon kirjoittamaa gradu3-dokumenttiluokkaa.
%
% Jos kirjoitat pro gradu -tutkielmaa, tee mallipohjaan seuraavat muutokset:
%  - Poista dokumenttiluokasta optio bachelor .
%  - Poista makro \type .
%  - Lisää suuntautumisvaihtoehto makrolla \studyline .
%  - Lisää tieto ohjaajasta makrolla \supervisor .

\documentclass[utf8,bachelor,manualbib]{gradu3}

\usepackage{palatino} % valitaan oletusfonttia hieman tyylikkäämpi fontti

\usepackage{graphicx} % tarvitaan vain, jos halutaan mukaan kuvia
\usepackage{amsmath}  % tarvitaan käytettäessä monimutkaisten matemaattisten kaavojen ja \eqref-kaavaviittauksen yhteydessä
\usepackage{url} % tarvitaan \url-komentoa varten
\usepackage{booktabs}

% Otetaan käyttöön author-date-järjestelmän mukaiset lähdeviittaukset:
\usepackage{natbib}
% Vaihdetaan kirjoittajan nimen ja vuosiluvun väliseksi erottimeksi
% välilyönti (oletuserottimena on pilkku):
%\bibpunct{(}{)}{;}{a}{}{,}


% HUOM! Tämän tulee olla viimeinen \usepackage koko dokumentissa!
\usepackage[bookmarksopen,bookmarksnumbered,linktocpage]{hyperref}

%\addbibresource{viite.bib}% Lähdetietokannan tiedostonimi
%http://www.tex.ac.uk/tex-archive/macros/latex/exptl/biblatex-contrib/biblatex-chicago/latex/biblatex-chicago.sty
%http://www.tex.ac.uk/tex-archive/macros/latex/contrib/etoolbox/etoolbox.sty
%http://mirrors.med.harvard.edu/ctan/macros/latex/contrib/biblatex/latex/biblatex.sty
%http://ctan.mackichan.com/macros/latex/contrib/biblatex/latex/biblatex2.sty
%http://mirror.hmc.edu/ctan/macros/latex/contrib/logreq/logreq.sty
%https://github.com/Martin-Rotter/qt-survival-guide/blob/master/logreq.def

\begin{document}

\title{Keinonäkö- ja parannellun näön järjestelmien käyttöönotto uuden sukupolven ilmailussa}

\translatedtitle{Use of Synthetic and Enhanced Vision Systems in Next Generation Aviation}

%\studyline{}
\avainsanat{SVS, EVS, ilmailu, näkyvyys, HUD, HDD, lentäminen, lentoturvallisuus, IFR, NextGen, NGATS, EVO}
\keywords{SVS, EVS, synthetic vision, enhanced vision, aviation, HUD, HDD, flying, safety, visibility, IFR, NextGen, NGATS, EVO}
\tiivistelma{Tähän tulee tiivistelmä (tausta, tavoite, tulokset, johtopäätökset).
}
\abstract{Tähän tulee englanninkielinen versio tiivistelmästä.
}

\author{Matias Laitinen}
\contactinformation{\texttt{matias.laitinen@gmail.com}}
% jos useita tekijöitä, anna useampi \author-komento
%\supervisor{TODO}
% jos useita ohjaajia, anna useampi \supervisor-komento
%\type{bachelor} % tämän makron oletus on "pro gradu -tutkielma" ja bachelor-optiolla kandidaatintutkielma

\maketitle
  
\mainmatter

\chapter{TODO Johdanto}

Viime aikoina lentoturvallisuus on noussut useasti esille medioissa, kun sekä harraste- että kaupallisen ilmailun puolella on tapahtunut monenlaisia lento-onnettomuuksia tai ilmailun vaaratilanteita. Nämä onnettomuudet aiheutuvat useimmiten inhimillisistä virheistä. Tutkimuksen tavoitteena on ottaa selvää, millä eri tavoin olisi keinotekoisia näköjärjestelmiä käyttämällä mahdollista ehkäistä lento-onnettomuuksia ja parantaa lentäjän tilannetietoisuutta, etenkin huonon näkyvyyden olosuhteissa. Tällaiset järjestelmät ovat olleet sotilaspuolen käytössä jo pitkän aikaa, mutta siviili-ilmailussa niitä hyödynnetään vasta melko vähän. Kartoittamalla näiden järjestelmien kustannuksia ja käytettävyyttä saadaan toivottavasti tehtyä jonkinlaisia johtopäätöksiä niiden soveltuvuudesta käytäntöön. TODO: Kerro paremmin ja laita viitteet kuntoon.

\emph{TODO:}
The United States air transportation system is undergoing a transformation to accommodate a projected 3-fold
increase in air operations by 2025.1 Technological and systemic changes are being developed to significantly
increase the capacity, safety, efficiency, and security for this Next Generation Air Transportation System
(NGATS). One of the key capabilities envisioned to achieve these goals is the concept of Equivalent Visual
Operations (EVO), whereby Visual Flight Rules (VFR) operational tempos and also, perhaps, operating
procedures (such as separation assurance) are maintained independent of the actual weather conditions. One
methodology by which the goal EVO might be attainable is to create a virtual visual flight environment for
the flight crew, independent of the actual outside weather and visibility conditions, through application of
Enhanced Vision (EV) and Synthetic Vision (SV) technologies. \citep{baileyym2007}

\emph{TODO:}
The NASA SVS project is based on the premise that better pilot SA during
low-visibility conditions can be achieved by reducing the steps required to build a
mental model from disparate pieces of data through the presentation of how the
outside world would look to pilots if their visibility was not restricted. \citep{prinzel2004}


\emph{TODO:}
NASA is conducting research, development, test, and evaluation of flight deck display technologies that may significantly enhance the flight crew's situation awareness, enable new operating concepts, and reduce the potential for incidents/accidents for terminal area and surface operations. The technologies that form the backbone of the BTV operational concept include: surface and airport moving maps; head-up and head-worn displays; four dimensional trajectory (4DT) guidance algorithms; digital data-link communications; synthetic and enhanced vision technologies; and traffic conflict detection and alerting systems (Bailey, Prinzel, Young, and Kramer, 2011; Prinzel et al., 2011). Preliminary research is described assessing a subset of these technologies in comparison to current-day low visibility surface operations. \citep{prinzel2013}


\chapter{Ilmailu ja näkyvyys}

\section{Näkyvyyden vaikutus lennon aikana}

Näkyvyys on lentokoneen ohjaajalle tärkeää lennettäessä lähellä maata, etenkin laskeutumisen aikana. Sen vuoksi huonot näkyvyysolosuhteet aiheuttavat suuria rajoitteita lentotoiminnalle \citep{mollersachs1994}. Ohjaajan lentonäkyvyyteen vaikuttavat monet meteorologiset olosuhteet, kuten pimeys, pöly, sumu ja sade \citep{wickens2009}. Erityisesti sumuisissa olosuhteissa näkyvyys voi huonontua voimakkaasti ja ulkomaailman yksityiskohtia on miltei mahdotonta erottaa \citep{beiergemperlein2004}.

Kaikilla lentokentillä toimittaessa ovat voimassa tietyt näkyvyysrajoitukset. Kentillä, joilla on käytössä esimerkiksi ILS:n kaltaisia lähestymisapuja, on tietty minimi, josta tulee olla mahdollista jatkaa lähestymistä visuaalisesti. Vakavimmillaan rajoitukset vaikuttavat kentillä, joilla ei tällaisia järjestelmiä ole käytössä. Nämä säännöt ovat voimassa, vaikka lähestyvällä koneella olisikin käytössään nykyaikaiset mittari- ja navigointilaitteet. \cite{mollersachs1994}.

Jo ilmailun alkuajoista asti on ilmailuteollisuus kehittänyt laitteita, joilla voittaa näitä huonon näkyvyyden rajoituksia. Tällaisia voivat olla esimerkiksi lentoasentojärjestelmät, navigointilaitteet, mittarilähestymislaitteet (Instrument Landing System, ILS), liikkuvat karttalaitteet sekä korkeasta maastosta varoittavat järjestelmät (Terrain Awareness and Warning System, TAWS). Näissä voidaan kuitenkin havaita se ongelma, että kaikki nykyaikaisetkin informaation esittämiseen tarkoitetut järjestelmät vaativat ohjaajilta jatkuvaa tiedonhakua ja -käsittelyä pysyäkseen selvillä ilma-aluksensa tulevista liikkeistä huonon näkyvyyden olosuhteissa \citep {prinzel2004}.

FSFN:n(Flight Safety Foundation) mukaan miltei 60 \% kaupalliseen lentotoimintaan liittyvistä maahansyöksyonnettomuuksista on tapahtunut lähestymisen tai laskeutumisen aikana. Etheringtonin ym. \citeyearpar{etherington2000} mukaan ohjattavissa olevan ilma-aluksen törmäys maastoon (Controller Flight Into Terrain, CFIT) on vallitseva onnettomuustyyppi ja vastuussa yli puolesta kaupallisen ilmailun onnettomuuksista. Myös Lentokonevalmistaja Boeingin \citeyearpar{boeing1996} tilastot tukevat tätä näkemystä. CFIT- onnettomuudet liittyvät yleensä paikka- tai asentotiedon (Situational Awareness, SA) menetykseen lähestymis- ja laskuvaiheessa, kun ohjaajat menettävät käsityksensä suunnasta, korkeudesta ja suhteesta ympäristöön \cite{schnellym2004}.

\section{Näkyvyyden vaikutus maatoimintaan}

Prinzel ym. \citeyearpar{prinzel2013} toteavat huonon näkyvyyden aiheuttamien toiminnan hidastumisten ja viivästysten maatoiminnassa olevan kasvavasti vaikuttamassa myös ilmatilankäytön viiveisiin. Huonon näkyvyyden olosuhteissa ohjaajien ja ajoneuvonkuljettajien tulee säilyttää tilannetietoisuutensa varmistaakseen, että maatoiminta on turvallista ja tehokasta. FAA:n mukaan \citeyearpar{gerold2001} Lennon vaarallisin vaihe onkin juuri maassa rullaaminen.

Esimerkiksi FAA:n vuoden 2010 turvallisuusselvityksen mukaan 52 928 316 maatoimintaan liittyvän tapahtuman aikana tapahtui 951 kiitotiepoikkeamaa (Runway Incursion), joista 12 oli vakavia. Vaikka tämä luku on suhteessa pieni, kiitotiepoikkeamalla voi olla tuhoisat seuraukset. Suurimpana syynä näissä tapauksissa oli ohjaajan inhimillinen erehdys (63 \%). Tilannetietoisuutta parantamalla voitaisiin siis saada merkittävästi vähennettyä kiitotiepoikkeamien määrää. \citep{prinzel2013}.

Tilannetietoisuutta maatoiminnassa voitaisiin parantaa esimerkiksi käyttämällä infrapunakameroita jopa tiheän sumun olosuhteissa, jotta havaittaisiin paremmin esteitä, kuten henkilöstöä, ajoneuvoja ja laitteistoja. \citep{beiergemperlein2004}.

\chapter{Järjestelmät}

\section{Keinonäköjärjestelmät (Synthetic Vision)}

Keinonäköjärjestelmillä (Synthetic Vision System, SVS, Synthetic Vision Information System, SVIS) tarkoitetaan keinotekoisen ympäristökuvan luomista  tietokoneella \citep{baileyym2007} tai NASA:n Rockwell Collinsin kanssa kehittämää tällaista teknologiaa käyttävää järjestelmää, SVS \citep{crawfordneal2006}. Prinzelin ym. \citeyearpar{prinzel2004} mukaan NASA:n ilmailuturvallisuusohjelman osana SVS:n tarkoituksena on eliminoida huono näkyvyys lento-onnettomuuksien aiheuttajana sekä parantaa yleis- liike- ja kaupallisen ilmailun operationaalisia valmiuksia.

Kuva luodaan yhdistäen lentoasento- ja tarkkuusnavigointijärjestelmiltä sekä maasto- ja estetietokannoista saatua lennon kannalta tärkeää tietoa. \cite{schnellym2004} toteavat SVIS -järjestelmien olevan uuden sukupolven ohjaamojärjestelmä, joka tulee olemaan tärkeässä osassa tulevaisuuden kaupallisten lentokoneiden ohjaamoissa. Crawford ja Neal kuitenkin \citeyearpar{crawfordneal2006} huomauttavat, että tieto ei välttämättä ole aina täysin ajan tasalla, koska se on peräisin tietokannasta. He mainitsevat myös tiedonkäsittelystä aiheutuvan viiveen voivan aiheuttaa hämmennystä, mikäli näkyvyyden palautuessa visuaalinen näkymä ei olekaan yhtenevä järjestelmän esittämän tiedon kanssa.

Suunnitellessa uusia ohjaamoita voidaan ottaa luonnollisesti huomioon SV-näyttöjen tarpeet (forward fit), mutta oman haasteensa asettaa se, miten käytännölliset ja toimivat näyttölaitteet voidaan asentaa vanhempiin ohjaamoihin, tai ohjaamoihin, joissa tilaa on rajallisesti (retrofit). Eräs ratkaisu tilankäyttöön olisi näyttää SVS heijastusnäytöllä, jolloin voitaisiin keskittyä siihen, kuinka parhaiten esittää keinotekoista kuvaa maastosta sen rajoitettujen grafiikkaominaisuuksien vuoksi. Toinen tapa olisi poistaa perinteiset mittarit ja asentaa tilalle keinonäkönäytöt, jolloin puolestaan jäisi selvitettäväksi, millaiset näyttöesitykset toimivat parhaiten miehistön näkökulmasta. \citep{prinzel2004}.

\section{Parannellun näön järjestelmät (Enhanced Vision)}

Parannellun näön järjestelmillä (Enhanced Vision System, EVS tai Enhanced Flight Vision System, EFVS) tarkoitetaan elektronisen apuvälineen, kuten lämpökameran (Forward-Looking Infrared, FLIR) tai millimetritutkan (Millimeter Wave Radar, MMWR) avulla näytettyä kuvaa ulkomaailmasta \citep{baileyym2007}. Möller ja Sachs \citeyearpar{mollersachs1994} toteavat optisten järjestelmien olevan passiivisia laitteita, joilla voidaan muodostaa ympäristökuva ilman etäisyystietoa. Sitä vastoin esimerkiksi tutkalla saadaan aktiivisesti etäisyystietoa ympäristöstä, mutta tavallisen näköistä kuvaa on vaikea muodostaa.

Kuten Crawford ja Neal \citeyearpar{crawfordneal2006} mainitsevat, voidaan EV -järjestelmiä käyttää yhdessä sekä heijastusnäyttöjen (Head Up Display, HUD), että ohjauspaneelin näyttöjen (Head Down Display, HDD) kanssa, tarjoten lentäjälle lämpökamerakuvaa maastosta ja liikenteestä myös heikoissa valaistus- ja sääolosuhteissa. \emph{TODO: (MAINITSE HDD)}

\section{Heijastusnäytöt (Head-Up Displays)}

SVS/EVS-tietoa voidaan esittää esimerkiksi heijastusnäyttöjen (Head-Up Display, HUD) avulla. HUD-näytöissä haluttu informaatio ja symbolit heijastetaan läpinäyvälle näytölle, josta ohjaaja katsoo läpi. Tällöin on mahdollista nähdä yhtä aikaa näytöllä oleva tieto sekä ulkona oleva näkymä, mikä mahdollistaa vähemmän aikaa käytettävän mittaritaulun tarkkailuun ja ympäristön seuraaminen on helpompaa. \citep{crawfordneal2006}

Ensimmäiset HUD:t olivat käytössä 1950-luvulla ja niitä käytettiin tähtäiminä, eikä pääasiallisesti lentomittaristona \citep{crawfordneal2006}. Ensimmäistä kertaa HUD oli käytössä koneen lentomittaritiedon näyttämiseen tarkoitettuna välineenä Hawker Siddeley Buccaneer -koneessa vuonna 1960 \citep{weintraubensing1992}. Tämä HUD koostui keinohorisontista sekä lentokoneen asentoa esittävästä symbolista. Korkeus ja nopeus näytettiin digitaalisesti ja ohjaintietokone näytti karkeaa opastustietoa. Nykyäänkin heijastusnäytöissä käytetään samankaltaista symbologiaa. HUD:a käytetään visuaalisena apuna kahdessa päätehtävässä: näkölähestymisessä sekä siirryttäessä mittarilento-olosuhteista visuaaliseen laskeutumiseen. HUD:n käytöstä päättää koneella operoiva lentoyhtiö. \citep{crawfordneal2006}

HUD:t ovat tulossa käyttöön myös yleisilmailun käyttöön osaltaan sotilas- ja siviilipuolella havaittujen etujen ansiosta \citep{ververswickens1998}. Crawford ja Neal \citeyearpar{crawfordneal2006} mainitsevat kuitenkin, ettei sotilaspuolen tutkimustuloksia voida helposti soveltaa sellaisenaan kaupalliselle sektorille erilaisten varusteiden ja lentotilanteiden vuoksi. Verrattuna saman informaation näyttämiseen näkökentän alapuolella sijaitsevassa perinteisessä mittaritaulussa tai HDD-näytöillä, voidaan HUD:lla tyypillisesti saavuttaa parempi suorituskyky etenkin seuraavilla alueilla:

\begin{itemize}
\item Vähemmän aikaa katse mittaritaulussa (head-down time) kriittisissä lennon vaiheissa ja vähemmän tarvetta tarkentaa katsetta läheltä kauas (mittareista ulkomaailmaan) \citep{maywickens1995}
\item Parempi lentoreitin säilyttäminen \citep{fischerym1980, lauberym1982, wickenslong1995}
\item Parempi tietoisuus ulkomailmasta ja odotettavissa olevien tapahtumien tai varoitusten havaitseminen ulkona tai näytöllä \citep{faddenym2000, fischer1979, larishwickens1991, maywickens1995, wickenslong1995}
\item Tarkemmat laskeutumiset \citep{naish1964}
\item Paremmat nousu- ja laskeutumisminimit joillakin lentokentillä ja ilma-alustyypeillä \citep{crawfordneal2006}
\item Parempi laatu mittaritiedon näyttämisessä \citep{maywickens1995}
\end{itemize}

\section{Maatoiminnassa saavutettavia etuja}

Maatoiminnassa lennonjohdon, koneiden ohjaajien sekä ajoneuvonkuljettajien tilannetietoisuutta pyritään pitämään yllä tarjoamalla visuaalisia merkkejä omasta sijainnista, kulkureiteistä ja tilasta kiito- ja rullausteillä, odotuspaikoilla ja asematasoilla. Tämä hoidetaan esimerkiksi valojen, merkintöjen ja opasteiden avulla. Tällaisia järjestelmiä kutsutaan nimellä Surface Movement Guidance and Control System (SMGCS). \citep{prinzel2013}.

Maatoiminnan tilannetietoa ylläpitäviä järjestelmiä voitaisiin myös käyttää ohjaamoissa. Tällaisista järjestelmistä voisi Prinzelin ym. \citeyearpar{prinzel2013} mukaan olla hyötyä, varsinkin miehistön näkyvyyden parantamisessa keinotekoisesti sekä tilannetietoisuuden (paikka- ja reittitiedon ja mahdollisesti myös liikenne- ja estetiedon) parantamisessa erilaisten karttajärjestelmien avulla. Etenkin yöllä, tai savun tai pölyn haitatessa näkyvyyttä EV-järjestelmät voivat auttaa ohjaajia toimimaan turvallisemmin maassa \citep{prinzel2013}. Hooeyn \& Foylen \citeyearpar{hooey2007} tutkimuksen mukaan 17\% yö- tai huonon näkyvyyden olosuhteissa tapahtuneessa rullauskokeessa tuli esille navigointivirheitä, jotka saatiin korjattua liikkuvien lentokenttäkarttojen (Airport Moving Map, AMM) avulla. Liikkuvat kartat parantaisivat huomattavasti ohjaajan tilannetietoisuutta, toteavat myös Möller ja Sachs \citeyearpar{mollersachs1994}.

Prinzelin ym. \citeyearpar{prinzel2013} tutkimuksen perusteella pelkkä EV-järjestelmien käyttö ilman karttaa ei merkittävästi paranna toimintaa maassa. Myös ohjaajat arvioivat AMM -karttojen olevan olennaisessa osassa maatoimintaa ja yhdessä EV sekä AMM toivat huomattavia etuja, esimerkiksi mahdollistamalla toiminnan entistä huonomman näkyvyyden olosuhteissa kentillä, joilla on vain vähän tai ei lainkaan rullausapuja käytössä. \citep{prinzel2013}.

\section{Lentotoiminnassa saavutettavia etuja}

Bennet ja Flach \citeyearpar{bennetflach1994} väittävät tiedon näyttämisen ohjaamoissa dynaamisten ja graafisten esitysten avulla johtavan ihmiskeskeisempään malliin, jossa jatkuva tiedonsaanti näyttäisi luonnollisen kaltaiset rajat turvalliselle toiminnalle. Tällöin korostuisi ihmisen joustavuus käyttää luonnollista ja koodattua informaatiota parhaiten hyväkseen.

Luonnollisella informaatiolla käsitetään tietoa, jota saadaan samalla tavoin kuin näkölento-olosuhteissa katsomalla ulos ohjaamosta. Koodattu informaatio sen sijaan vaatii ohjaajalta erikseen sen todellisen arvon ymmärtämistä. Luonnollista informaatiota voi olla esimerkiksi korkeuden silmämääräinen arviointi ja koodattua informaatiota korkeusmittarin näyttämä. \citep{prinzel2004}. SV -järjestelmien avulla voidaan esittää tällaista luonnollisen kaltaista ja intuitiivista informaatiota, jota on helppo käsitellä \citep{wickensandre1990}.

SVS-teknologia voi mahdollistaa rajoitetusta näkyvyydestä aiheutuvien ongelmien ratkaisemisen visuaalisesti, tehden lentotoiminnasta säästä riippumatta samanlaista kuin kirkkaassa päivänvalossa ja parantaen ohjaajien tilannetietoisuutta \citep{prinzel2004}. Koska rajoittunut näkyvyys on suurimpana tekijänä monissa kohtalokkaissa lento-onnettomuuksissa \citep{boeing1996}, voisi SVS:n käytöllä olla merkittävä vaikutus turvallisuuteen. Jo pelkät lentoturvallisuuden hyödyt, joita SVS mahdollistaa, ovat tarpeeksi aiheen tutkimiselle, mutta koska kyseinen järjestelmä on hyvin kallis, on löydettävä myös operationaalisia ja taloudellisia hyötyjä \citep{prinzel2004}.

Seuraavanlaisia hyötyjä ainakin NASA \citeyearpar{williamsym2001} arvioi voitavan saavuttaa kasvavan lentoliikenteen myötä esimerkiksi seuraavien etujen kautta:

\begin{itemize}
\item Pienempi ajankäytön tarve kiitotiellä huonon näkyvyyden olosuhteissa
\item Pienemmät lähtö - ja tulominimit
\item Helpommin sallittavat erilaiset lähestymistavat, etenkin rinnakkaisille kiitoteille
\item Pienemmät porrastukset saapuvien ilma-alusten välillä
\item Toisistaan riippumattoman toiminnan mahdollistaminen rinnakkaisilla, lähekkäin sijaitsevilla kiitoteillä
\end{itemize} 

Myös Schnellin ym. \citeyearpar{schnellym2004} mukaan SVIS-järjestelmät antavat ohjaajille tehtäväkohtaista tietoa ja opastusta, jota tarvitaan lennettäessä monimutkaisia, kaartuvia lähestymispolkuja. Lisäksi hekin mainitsevat nykyisen ilmatilan olevan varsinkin joillain lähestymisalueilla kapasiteettinsa rajoilla. Jo pieniki muutos säätilassa tai laitteisto-ongelmat lentokentällä voivat aiheuttaa liikenteen ruuhkautumisen. SVIS-järjestelmistä kaavaillaan mahdollista ratkaisua tähän.

\cite{baileyym2007} väittävät SV-järjestelmillä saatavan huomattavia parannuksia maastoestetietouteen ja että se voisi vähentää CFIT-onnettomuuksien riskiä nykyiseen ohjaamoissa käytettävään teknologiaan verrattuna. Tästä samaa mieltä vaikuttavat olevan myös \cite{schnellym2004} todetessaan, että SVIS voisi olla avainteknologia CFIT-onnettomuuksien vähentämisessä.

Tutkijat selvittävät SV- ja EV-järjestelmien yhdistämismahdollisuuksia (Enhanced Synthetic Vision System). Tällainen järjestelmä voisi antaa ohjaajalle liikennetietoa samalla tavoin kuin lennettäessä selkeässä säässä päivänvalossa. Esitettäessä paljon tietoa eri järjestelmistä samalla näytöllä, on otettava huomioon inhimilliset tekijät, lentäjän suorituskyky, ongelmat ja turvallisuusnäkökohdat. \citep{crawfordneal2006}

SV-EV yhdistäminen, tähän kuva \citep{mollersachs1994}

Nordwallin \citeyearpar{nordwall1993} mukaan heijastusnäytön ja lämpökameran yhdistelmällä saavutetaan esimerkiksi sumussa huomattavasti parempi kuva ynpäristöstä, kuin mitä paljaalla silmällä voitaisiin. Beierin ja Gemperleinin \citeyearpar{beiergemperlein2004} tutkimuksen perusteella aivan tiheässä sumussa, jossa näkyvyydet ovat 300m tai vähemmän, ei lämpökameroita käyttämällä kuitenkaan saada parannusta näkyvyyteen. Myös Crawford ja Neal \citeyearpar{crawfordneal2006} toteavat kovan sumun, sateen tai pölyn heikentävän lämpökameralla saavutettavia etuja. Tällaisissa tapauksissa lähestymisessä tarvittaisiin esimerkiksi tutkajärjestelmä ympäristökuvaa luomaan.

Baileyn ym.\citeyearpar{baileyym2007} mukaan näiden teknologioiden optimaalisin yhdistelmä olisi näyttää ohjaamomiehistölle suoraan keinonäköjärjestelmän informaatiota, mutta EV-järjestelmä toimisi ikäänkuin taustalla, suorittaen navigaatiovirheiden korjausta, tietokannan eheyden valvontaa sekä reaaliaikaista esteentunnistusta. Kuvanprosessointi toimisi automaattisesti taustalla ja voitaisiin näyttää ohjaajille tietoa säästä rippumatta.

Schnell ym. \citeyearpar{schnellym2004} lisäävät, että SVIS:n käyttö yhdessä GPS-navigointijärjestelmien kanssa voisi mahdollistaa uusia jatkuvan laskun kaartolähestymismenetelmiä, jotka olisivat nykyisiä suoria ja portaittaisia lähestymisiä tehokkaampia. Tällaisia lähestymismenetelmiä miehistön on vaikeaa suorittaa perinteisillä lentonäyttöjärjestelmillä. Etenkin ajallisesti kaartolähestymisiä on haastavaa käsittää verbaalisesti tai paperilla. SVIS helpottaisi asiaa ilmassa sijaitsevalla keinotekoisella, kolmiuloitteisella tunnelilla, jota on helppo seurata \citep{barrowspowell1999}.

Ehkä kuvia tähän vielä \citep{schnellym2004} \emph{TODO:}

\chapter{DO Ohjaajan suorituskyky ja kognitiiviset haasteet}

Ihmisen suorituskyky aiheuttaa paljon haasteita, kun kehitetään uusia tapoja 
esittää tietoa ohjaamomiehistölle. \emph{TODO LAINAA KAIKKI JA KEKSI JOHTADANTO Seuraavaksi käsitellään joitakin ihmisen suorituskyvystä aiheutuvia haasteita, joita keinonäkö- ja parannellun näön järjestelmiä kehiteltäessä tulisi ottaa huomioon.}

\section{Tilannetietoisuus (Situational awareness)}

Hyvän tilannetietoisuuden ylläpitäminen on erittäin olennaista turvallisen lennon kannalta. Schnell ym. \citeyearpar{schnellym2004} toteavat tilannetietoisuuden olevan menetetty, mikäli ohjaamomomiehistö ei osaa vastata seuraaviin kysymyksiin:

\begin{itemize}
\item Missä ollaan?
\item Minne ollaan menossa?
\item Mitä tehdä, kun päästään sinne?
\end{itemize}

Lennettessä mittarilahestymisiä haastavissa olosuhteissa on tärkeintä tietää oman tilankäytön vaatimukset ja maaston sille asettamat haasteet. Lisäksi Schnell ym. \citeyearpar{schnellym2004} mainitsevat tilannetietoisuuden voivan olla myös ajallista, sillä miehistön tulisi olla jatkuvasti selvillä siitä, mitä tehtäviä on suoritettava missäkin vaiheessa lentoa.

Tilannetietoisuuden kadotessa hetkellisesti voidaan joutua epätavalliseen lentotilaan. Tällöin on olennaista, kuinka nopeasti saadaan tilannetietoisuus palutettua ja virheellinen lentoasento saadaan oikaistua tehokkaasti. Käytettäessä perinteisiä mittareita tai ohjainpaneelissa sijaitsevaa näyttölaitetta, voidaan helposti värien avulla kertoa ohjaajalle, missä päin maa ja taivas sijaitsevat. Sen sijaan ainakin nykyiset monokromaattiset heijastusnäytöt aiheuttavat haasteita, kun lennetään mittarilento-olosuhteissa ja joudutaan yllättäen epätavalliseen lentotilaan.

Newman \citeyearpar{newman2000} on listannut HUD:en ominaisuuksia, jotka saattavat vaikeuttaa oikaisua epätavallisista lentotiloista:

\begin{itemize}
\item Merkistön sekavuus (Clutter)
\item Näytön kehykset (Framing)
\item Silmän tarkentumista häiritsevät tekijät (Accommodation traps)
\item Ylösalaiset symbolit
\item Digitaalisen tiedon esittäminen
\item Todellisen kokoiset pituuskallistusmerkinnät
\item Näytön käyttäytyminen ohitettaessa suoraan ylös (zenith) tai alas (nadir) -merkinnät
\item Nopeusvektorin hallinta
\end{itemize} 

Digitaalisessa muodossa olevan tiedon hahmottaminen nopeasti voi olla hankalaa, mutta analogiset nauhat ja osoittimet voivat lisätä heijastusnäytön sekavuutta \citeyearpar{zuschlag2003}. Todellisen kokoiset pituuskallistusmerkinnät voivat nopeasti edetessään olla myös vaikeita hahmottaa. Newmanin \citeyearpar{newman1995} mukaan tällaisissa tilanteissa pituuskallistusmerkintöjen tiivistäminen hidastaisi niiden liikettä ja saada ohjaajan tunnistamaan paremmin epätavalliseen lentotilaan joutumisen. Mikäli heijastusnäytön nopeusvektorisymbolia käytetään hallitsevana havaintotekijänä voidaan joutua tilaan, jossa todellinen kohtauskulma on ylempänä kuin lentosuunta ja tällöin on virheellistä vetää sauvasta, vaikka nopeusvektori onkin matalalla \citep{crawfordneal2006}. FAA on ohjeistanut näyttöjen suunnittelijoita ottamaan huomioon, että mikäli heijastusnäytöllä käytetään erityistä merkistöä ohjeistamaan oikaisussa vaadittavia ohjainliikkeitä, tulisi käyttää sellaisia symboleja, jotka eivät sekoitu tavanomaisesti näkyvissä oleviin indikaattoreihin \citep{crawfordneal2006}.

Monista nykyaikaisista varoitujärjestelmistä poiketen SVS-järjestelmät voisivat osaltaan jo ennaltaehkäistä vaarallisiin tilanteisiin joutumista, sen sijaan että toimivat vasta tilanteen sattuessa. \citep{schnellym2004}. Newman \citeyearpar{newman2000} mainitseekin heijastusnäyttöjen käyttämisessä saavutettavat hyödyt ylittävät niiden mahdolliset haittavaikutukset.

\section{Huomiokyvyn kaventuminen (Attentional tunneling)}

HUD-näyttöä käytettäessä voisi olettaa, että ohjaajan on helpompi havaita ulkona tapahtuvia asioita, kun katsetta ei tarvitse siirtää ohjaintaulusta ylös havaintojen tekemiseksi. Useat tutkimukset \citep{fischerym1980, weintraubensing1992, wickenslong1995, wickensalexander2009} kuitenkin osoittavat, että heijastusnäytön symbolit voivat kiinnittää liikaa ohjaajan huomiota puoleensa ja heikentää hänen kykyään havainnoida ennalta odottamattomia ympäristön tapahtumia. Samankaltaisesta ilmiöstä on kyse, kun esimerkiksi puhutaan puhelimeen autolla ajaessa, mikä voi johtaa huomion keskittymiseen keskusteluun ajamisen sijaan \citep{horreywickens2006, strayerdrews2007, strayerym2001}. Wickens ja Alexander \citeyearpar{wickensalexander2009} toteavat, että näkökentän sisälläkään olevat objektit eivät automaattisesti herätä tarpeeksi huomiota ja määrittelevät huomiokyvyn kaventumisen (attentional tunneling): \emph{"The allocation of attention to a particular channel of information, diagnostic hypothesis, or task goal, for a duration that is longer than optimal, given the expected cost of neglecting events on other channels, failing to consider other hypotheses, or failing to perform other tasks"}.

Huomiokyvyn kaventumisen voimakkuuteen vaikuttaa huomiota vaativien tehtävien luonne, sekä tapa, jolla tietoa aistitaan. Selkeän sään olosuhteissa ohjaajan huomiokyky keskittyy selvästi enemmän ulos, kun oikea horisontti on paremmin näkyvissä ja näin ollen heijastusnäytön symboleita suurempana tarjoaa paremman näyttämän asentotiedosta \citep{ververswickens1998}. Crawford ja Neal \citeyearpar{crawfordneal2006} toteavat ympäristön kanssa yhtenevien (conformal) HUD:n symboleiden helpottavan ymmärtämistä. Wickensin ja Hollandsin mukaan \citeyearpar{wickenshollands2000} ihmisen onkin helpompi havaita ympäristössään tapahtuvia asioita, mikäli heidän huomionsa on keskittynyt alueelle, jossa tapahtumat esiintyvät. Esimerkiksi samankaltaisten objektien ryhmitteleminen saattaa tukea huomiokyvyn jakautumista, mutta vaikeuttaa keskittymistä yhteen tiettyyn asiaan näytöllä. Samalla tavoin dynaamiset, liikkuvat kohteet voivat herättää erittäin paljon huomiota muilta visuaalisilta elementeiltä \citep{crawfordneal2006}.

Ohjaajan huomiokykyä tutkittaessa tulee ottaa huomioon, suunnitellaanko tutkittavia järjestelmiä käytettävän yhden vai useamman ohjaajan miehistöllä, sillä molemmat ohjaajat useinkaan keskity lennon aikana samojen työtehtävien hoitamiseen, vaan jakavat niitä keskenään, ja esimerkiksi ulkona sijaitsevien odottamattomien tapahtumien havainnointi voi kuulua enemmän toisen ohjaajan työtehtäviin \citep{crawfordneal2006}. Kehitettäessä järjestelmiä yleisilmailun tarpeisiin tulisi huomiokyvyn kaventumisen tutkimuksissa painottaa sitä, kuinka HUD:n käyttö vaikuttaa ulkona esiintyvien tapahtumien havainnointiin VFR-olosuhteissa, kun ohjaajalla on päävastuu riittävän erotuksen säilyttämisestä \citep{ververswickens1998} sekä ohjaajan huomiokykyä toimittaessa yksin ohjaamossa \citep{crawfordneal2006}. 

Mikäli HUD- tai HDD-näyttöä käytetään keinonäköjärjestelmän tuottaman kolmiulotteisen kuvan näyttöön, saattaa ohjaaja uppoutua aidontuntuiseen 3D-näkymään (3D immersion) ja jättää muiden näyttöjen näyttämää tietoa tai ulkopuolella sijaitsevia tapahtumia huomioimatta \citep{olmosym2000}. Tästä on luonnollisesti haittaa silloin, mikäli ulos katsomalla on saatavilla sellaista olennaista tietoa, mitä keinonäköjärjestelmän näyttämä ei sisällä \citep{foylehooey2003}. Crawfordin ja Nealin \citeyearpar{crawfordneal2006} tutkimukset osoittavat, että HUD:n käyttö voi nopeuttaa odotettavissa olevien tapahtumien havainnointia näytöllä, mutta hidastaa sekä lähellä että kaukana esiintyvien, odottamattomien tapahtumien huomaamista. Tämä ero on huomattava etenkin työkuormituksen (workload) kasvaessa \citep{larishwickens1991}.

On todennäköistä, että ohjaajalle odottamattomien tapahtumien määrä vähenee yhdistettyjen SV- ja EV-järjestelmien kehittyessä yhä luotettavammiksi \citep{kornym2009}. Tämä saattaa kuitenkin lisätä luottamusta järjestelmiin liikaa, ja heikentää ennestään sellaisten tapahtumien huomaamista, joista järjestelmä ei syystä tai toisesta pysty ilmoittamaan \citep{molloyparasuraman1996}. Tällaisen automaatiovinouman \citep{mosierym1998} vaikutuksen huomiokyvyn kaventumiseen mainitsevat myös Wickens ja Alexander \citeyearpar{wickensalexander2009}. Seuraavaksi käsitellään hieman tarkemmin automaatiovinoumaa ja sen vaikutuksia.

\section{Automaatiovinouma (Automation bias)}

Osa kognitiivisista lennonaikaisista tehtävistä (reitinlaskenta, navigointi, järjestelmien vikailmoitukset) voidaan hoitaa automatiikan avulla tai päätöksentekoa helpottavien työkalujen avustuksella. Koska ohjaamoympäristössä käsitellään entistä enemmän ja monimutkaisempaa tietoa, on odotettavissa, että ohjaamoautomatiikan määrä lisääntyy nopeasti uuden sukupolven ohjaamoissa. Vaikka tiedon käsittelyn ja varastoinnin tehostuminen onkin hyödyksi, automatiikkaan tottuminen muuttaa käyttäjäkokemusta ja toimintaa sekä nostaa esiin uudenlaisen käyttäjäongelman, automaatiovinouman (automation bias). Automaatiovinoumalla tarkoitetaan virheitä, jotka aiheutuvat siitä, että käyttäjä luottaa liikaa automaatioon ja korvaa automaattisten järjestelmien luomilla merkeillä muun aktiivisen tiedonhakunsa ja -käsittelynsä. \citep{mosierym1998}.

Ihmiselle on luonnollista pyrkiä toimimaan siten, että hän joutuu käyttämään mahdollisimman vähän vaivaa kognitiiviseen työskentelyyn ja usein erilaiset oikopolut ja heuristiikat tarjoavat siihen parhaan mahdollisuuden \citep{fisketaylor1994}. Tämä houkutteleekin käyttämään automaattisia merkkejä heuristisesti muun saatavilla olevan informaation sijaan. Mosierin ym. \citeyearpar{mosierym1994} mukaan automaatiovinouman aiheuttamat laiminlyöntivirheet ovat todennäköisimpiä matkalentovaiheen aikana, kun ohjelmoidaan jokin järjestelmä suorittamaan tehtävää ja luotetaan täysin siihen, tarkastelematta muita merkkejä, jotka saattaisivat viestiä epätavallisuuksista tai vikatilanteista. Käyttäjän vastuuntuntoisuus ja kokemus vaikuttavat huomattavasti siihen, tarkistaako hän automatiikan antamien merkkien oikeellisuuden muita menetelmiä käyttäen, vai jättääkö ne huomiotta \citep{mosierym1998}.

\section{Näyttöjen sekavuus (Display clutter)}

Tehokkaan ohjaamotyöskentelyn kannalta on tärkeää, että vain tehtävän kannalta oleellisin tieto näytetään ohjaajalle, jotta sitä on helpompi hahmottaa ja käsitellä \citep{ververswickens1998}. Muuttuvin tilanteisiin on kyettävä reagoimaan nopeasti ja liiallinen informaation määrä voi vaikeuttaa tiedon etsintää. Piirrettäessä tietoa HUD:lle on myös vaarana, että liialliset symbolit peittävät näkyvyyttä ulos. Crawford ja Neal \citeyearpar{crawfordneal2006} toteavat sekavuuden aiheuttavan huomiokyvyn kaventumista, jota käsiteltiin tässä tutkielmassa tarkemmin hieman aiemmin. Myös Ververs ja Wickens \citeyearpar{ververswickens1996} mainitsevat näytön sekavuuden (clutter) eliminoivan joitakin HUD:sta saatavia hyötyjä. Sekavuus saattaa vaikeuttaa tehtävän kannalta tärkeän tiedon havainnointia \citep{nikolicym2004, stelzerwickens2006, wickensym2003}.

Sekavuutta voidaan vähentää erottelemalla SV- ja EV -järjestelmistä tulevaa tietoa \citep{baileyym2007}:

\begin{itemize}
\item Tiedon sijainnin mukaisesti (spatial separation).
Tämä onnistuu käyttämällä eri näyttöjä, jolloin ohjaaja itse erottaa tarvitsemansa tiedon niistä. HUD:lta on löydyttävä tällöin kaikki tehtävän kannalta kriittinen informaatio, jotta mahdollisimman vähän aikaa  tarvitsee käyttää sen etsimiseen katse mittaritaulussa (head-down). Muutoin työkuormitus voi lisääntyä liikaa tai tilannetietoisuus voidaan menettää.
\item Ajallisesti (temporal separation).
Tässä tapauksessa automaattisesti ajastettu vaihtelu eri tiedonlähteiden välillä on havaittu ohjaajalle manuaalista luonnollisemmaksi.
\end{itemize}

Sekä näytön symboleissa että ympäristössä esiintyvien tapahtumien huomaaminen on helpompaa käytettäessä heijastusnäyttöä (head-up), kuin mittaritaulun näyttöä (head-down). Tähän vaikuttavat todennäkösesti HUD:n symboleita selkeyttävät paremmat kontrastierot, HDD:n visuaalisen tiedon mukauttamisen tarpeesta aiheutuva viive sekä HUD:n sijainnista johtuva vähäisempi tiedonetsinnän tarve. Vaikka näytön sekavuus voikin hidastaa mittarien ymmärtämistä ja ulkona esiintyvien kohteiden huomaamista, ovat HUD:n käytössä saavutettavat hyödyt silti siitä aiheutuvia haittoja suurempia. \citep{ververswickens1998}

Ververs and Wickens \citeyearpar{ververswickens1996} ovat havainneet myös tehtävän kannalta vähemmän tärkeän tai häiritsevän tiedon himmentämisen auttavan etenkin matkalentovaiheessa ohjaajia reagoimaan nopeammin suunnan, korkeuden ja nopeuden muutoksiin HUD:lla. Kuitenkin myöhemmässä tutkimuksessaan \citeyearpar{ververswickens1998} he rajaavat, että ainoastaan HDD:lla tehtävän kannalta hyödyttömän tiedon himmentämisestä olisi merkittävää hyötyä.

Näytettäessä keinonäkö- ja parannellun näön järjestelmien tuottamaa tietoa yhdistetysti, pienenee tarve siirtää katsetta etsittäessä tietoa eri lähteistä. Kuitenkin, tämä saattaa vaikeuttaa tiedon alkuperän hahmottamista ja ymmärtämistä, lisäämällä näytön sekavuutta. Tarpeettoman tiedon poistamisen hallinta (declutter control) mahdollistaa ohjaajan valita, mitä saatavilla olevasta tiedosta renderöidään näytölle, mutta sen käyttö voi kasvattaa työkuormitusta kriittisissä tilanteissa tai jopa tärkeän tiedon tahattoman näyttämättä jättämisen. \citep{baileyym2007}.

\section{Akkommodaatiovääristymä (Misaccommodation)}

Silmän akkommodaatiovääristymäksi kutsutaan tilannetta, jossa katse ja katsojan huomio tarkentuu jokin lähietäisuudellä sijaitsevaan kohteeseen ja vaikeuttaa muiden kohteiden huomiointia sekä koon ja etäisyyden arviointia. Tämän välttämiseksi heijastusnäytöt kohdistetaan usein optiseen äärettömyyteen, jotta näyttäisi, että näytön symbologia sijaitsisi samalla etäisyydellä ulkomaailman kanssa ja näin ollen vähentäisi katseen uudelleentarkentamisen tarvetta \citep{naish1964}.

HUD:en kohdistamisen tehokkuudesta ja tärkeydestä vat väitelleet esimerkiksi Newman \citeyearpar{newman1995} sekä Weintraub ja Ensing \citeyearpar{weintraubensing1992}. Koska erilaiset sääolosuhteet ja HUD:n tarkkuuden taso vaikuttavat huomion kohdistumiseen, ei ole vielä täysin varmaa, vähentävätkö kauas kohdistetut HUD:t akkommodaatiovääristymiä, vai lisäävätkö ne niitä \citep{crawfordneal2006}.

\chapter{TODO SV- ja EV-järjestelmien tulevaisuus}

\section{TODO Tulevaisuuden sovelluksia}

Tässä pohditaan, minkälainen tulevaisuus SV- ja EV-järjestelmiä mahdollisesti odottaa uuden sukupolven ilmailun näkökulmasta.

\emph{TODO:}
Based on our experimental data, it seems that SVIS displays such as our Condition
2 may find application in situations where the precision of aircraft navigation is of
utmost importance. Such situations could include commercial airline approaches
into terrain-challenged airports, where curved paths may be more effective in
maintaining maximum terrain separation than would be possible with conventional
approaches. Another application may be in situations where closely spaced
parallel approaches are conducted to parallel runways with minimal lateral separation. \citep{schnellym2004}

The primary key to
providing SVIS for application in the real world seems to be a refined guidance
concept with a carefully tuned flight path predictor and a guidance cue that is tuned
to the predictive nature of the flight path predictor. The depiction of the terrain is
probably a secondary key point that adds to the usefulness of the SVIS. \citep{schnellym2004}

\emph{TODO:}
EMERGING ISSUES AND APPLICATIONS
There is an abundance of new avionics available now and these are provided by
a multitude of manufacturers. For example, in 2002 Boeing conducted demonstrator
flights in a specially modified 737–900. The advanced avionics to be
evaluated by representatives from a variety of airlines included synthetic vision
system (SVS) displays; a head-up guidance system married to two infrared enhanced
vision systems (EVS); highway-in-the-sky navigational cues; a virtual-
traffic-cone surface guidance system; global positioning satellite (GPS)
landing system; and a software upgrade to the enhanced ground proximity warning
system (EGPWS). Other prevention technologies include NASA’s Runway
Incursion Prevention System, which alerts pilots on approach to other aircraft
that pose a threat. In the following sections, we consider the implications of
these types of emerging technologies for the use of HUDs.
Pathway HUDs
Pathway, tunnel, or highway-in-the-sky 3–D format flight path displays provide
a prediction and preview of a flight path. Although these displays have been in-vestigated for a few decades, combining the pathway display and HUD is a relatively
new concept.
Pathway HUDs have been investigated by Ververs and Wickens (1998) and
Fadden, Ververs, and Wickens (2000). The three elements that make up the new
HUD are a preview tunnel of where the aircraft will be in the future, a predictor
symbol, and a 3–D perspective. Preliminary tests with pilots showed positive performance
results, in the form of more precise tracking with lateral and vertical error
limited to approximately 10 ft. However, when an unexpected event arose for
pilots who were truly naive to the study (in this case a runway incursion), the event
was detected more slowly in the HUD condition than in the HDD condition. It
should be noted that this finding was not statistically significant, possibly due to
the insufficient power of the study. Nevertheless, the results are consistent with the
cognitive tunneling effects reported previously, without pathway displays. \citep{crawfordneal2006}



In addition, research is
warranted into the use of HUDs in single and multicrew environments; the effects
of pilot qualifications, training, and experience; and the training practices
and experience of operators.
Research is currently being conducted at the FAA/Volpe National Transportation
Safety Center to provide the FAA with guidelines for certifying HUDs for civilian
use (FAA, 2002). Twenty-two HUD design issues have been identified by
FAA experts while certifying HUDs. Further research is being conducted to determine
how pilot performance is affected by each of the design issues. The 22 issues
are broken down into the following categories: location and format design of flight
information, display effectiveness to support the intended task,HUDeffectiveness
in displaying and guiding recovery from unusual attitudes, consistency, and
discriminability ofHUDsymbology, and pilot physiological stress associated with
HUD optical design.
A further area of investigation might include HUDs and the proximity principle,
which is a principle of perception that was first identified by the Gestalt psychologists.
The central idea of this principle is that the smaller the gap between
stimuli, the more likely those stimuli are to be seen as belonging together. The gap
can be in terms of space or in terms of time. Although this principle has been investigated
by putting similar items together in the HUD visual field to ease processing,
the HUD could also break down the external field stimuli, making them hard
to interpret. Flight Lieutenant Robert Woodbury at RAAF Richmond demonstrated
this effect to the first author by showing a photographic image of an HUD
plus three lights visible in the external domain. It was not until the HUD was removed from the field of view that it was easy to see that the three lights were part of
a preceding aircraft. To our knowledge, no studies have investigated this issue. \citep{crawford2006}


They concluded that dividing attention between the two overlapping sources is a
difficult and unnatural cognitive task that may exhaust resources in
high-workload situations (Larish \& Wickens, 1991).

Pilots, on the other hand, appear to believe that HUDs reduce their workload.
In the Boeing study described earlier, the pilots reported that the HUD reduced
their workload.

Furthermore, the pilots reported that they found it easier to
switch attention from the HUD to the external scene, compared with the HDD,
and that the HUD was easier to use. The positive evaluations of the HUD provided
by the pilots occurred despite the fact that the HUD produced an increased
number of missed events.

These results suggest that the cognitive tunneling effect
is counterintuitive, and that many pilots are not aware of its existence. Hofer
et al. (2001) concluded that “Pilots think they are seeing everything because all
the information is being presented in their visual field when in fact they are not
attending and processing everything” (p. 2).

Additional studies into the effects
of workload on cognitive tunneling, and pilots’ awareness of these effects, need
to be carried out. \citep{crawfordneal2006}


\section{TODO NextGen ja EVO-konseptit}

\emph{TODO:}
Tässä osiossa kerrotaan NextGen-ilmailukonseptista sekä EVO-konseptista sekä näönparannusjärjestelmän vaatimuksista, joita näihin konsepteihin on visioitu.

\emph{TODO:}
Tämä saattaa olla hyvä laittaa edellisen osion alle (tai sitten ei), katsottava muotoiluvaiheessa.

\emph{TODO:}
NASA is striving to develop the technologies and knowledge to enable EVO and to extend EVO towards a “Better-Than-Visual” operational concept. This operational concept envisions an "equivalent visual" paradigm where an electronic means provides sufficient visual references of the external world and other required flight references on flight deck displays that enable Visual Flight Rules (VFR)-like operational tempos while maintaining and improving safety of VFR while using VFR-like procedures in all-weather conditions.\citep{prinzel2013}
Emerging flight deck technologies offer a potential means to create an equivalent level of safety and performance. These flight deck technologies, such as the E-SMGCS -AMM display and EV, could assist in fully realizing the potential of NextGen by offering a more affordable path toward safe and efficient LVO/SMGCS operations through an “equivalent visual” paradigm.\citep{prinzel2013}

\emph{TODO:}
Taken together, the conclusions that can be drawn from these experiments are that
synthetic vision can be implemented on retrofit sizes and, therefore, can successfully
be introduced into the current aircraft fleet. To be effective, synthetic vision
presented on small display sizes would have to be minified and our results indicate
that the MF should not exceed 4.5 for optimal performance, although more research
is needed to confirm such a conclusion.
Because no performance differences were found between photo-realistic and
generic terrain texture methods, the generic terrain presentation might represent an
effective and lower cost option for synthetic vision displays, although photo-realistic
terrain does have properties that can increase the margin for safety and operations.
Several pilots did comment that photo-realistic texture would be helpful for
SA during climb, en route, and descent phases of flight. A recent flight test in the
terrain-challenged area of Eagle-Vail, CO, however, found no performance or SA
penalties for the generic texture concept, although pilots reported an overall preference
for the photo-realistic presentation (Bailey et al., 2002; Prinzel et al., 2002).
The NASA Aviation Safety Program is currently evaluating a synthetic vision concept
that combines generic and photo-realistic terrain texture to take advantage of
the benefits both methods offer for SA. \citep{prinzel2004}

\emph{TODO:}
The problem of reduced visibility challenges aviation goals to reduce the accident
rate and improve operational capacity (Federal Aviation Administration, 2001;
NASA, 2001). The approach of synthetic vision is to solve the problem through the
presentation of how the outside world would look to the pilot if vision were not restricted.
TAWSare steps in the right direction and they have significantly improved
safety, but the solution treats the symptoms and not the cause (Moroze \& Snow,
1999). Synthetic vision instead provides for proactive prevention of visibility-induced
accidents while also increasing the capability to make approaches in
weather conditions and airports not currently legal for low-visibility operations.
Although our research did not specifically address these aviation safety and operational
benefits, subsequent studies (e.g., Prinzel et al., 2002) have substantiated the
performance and SA enhancements of synthetic vision even while making complex,
circling approaches under conditions that are beyond current cockpit technology
capabilities. Furthermore, the concept described here represents only the
database and display concepts and not the total SVS, which will include synthetic
vision navigation displays; runway incursion prevention technology; database integrity
monitoring equipment; enhanced vision sensors; taxi navigation displays;
and advanced communication, navigation, and surveillance technologies
(McCann et al., 1998; Timmerman, 2001; Uijt de Haag, Young, Sayre, Campbell,
\&Vadlamani, 2002;Williams et al., 2001; Young \& Jones, 2001). These technologies
represent a comprehensive solution that will be evaluated in near-term NASA
simulation and flight research. Together, synthetic vision may considerably help
meet national aeronautic goals to “reduce the fatal accident rate by a factor of 5”
and to “double the capacity of the aviation system,” both with 10 years (NASA,
2001, p. 2). \citep{prinzel2004}
























\chapter{TODO Tulokset ja Yhteenveto}

Synthetic vision has the potential to provide significant safety and economic benefits,
particularly if the system is effective as both a retrofit and forward-fit solution
to visibility-restricted problems. Previous research has shown the efficacy of synthetic
vision on large-size displays and, therefore, synthetic vision is expected to
be capable of effective presentation as glass displays become larger with each generation
of aircraft. Because the majority of the current commercial aircraft fleet has
electromechanical instruments or limited glass real estate, however, any significant
benefits would require answering the retrofit question of whether effective
presentation of synthetic vision can also be made in current aircraft cockpits.
Display Size
The retrofit question concerns our hypothesis that the HUD and smaller SVS display
sizes would provide adequate information to enable the pilot to make safe and
precise approaches. The results of the experiments confirmed the hypothesis and
suggest that SVS is viable as a retrofit candidate. Experiment 1 showed no differences
in path performance between display sizes or FOV, and Experiment 2
showed differences only for the HUD concept for lateral path performance. \citep{prinzel2004}

Analysis results indicate that the SVIS is superior to the conventional displays
for the majority of the measures that were obtained in the experiments.
Based on the results of the experiments, we feel that the SVIS is a display format
that improves the performance of pilots in many ways. \citep{schnellym2004}

In summary, research suggests that there are a number of advantages in using
HUDs. These include increases in flight path tracking accuracy, except during
cruise flight; benefits for event detection, except in the approach and landing
phase and for unexpected events; lower visibility takeoff and landing; more accurate
approach and landing; the elimination of head-down time; a reduction in
the time taken to refocus between instruments and the external scene; and the
potential to use overlaid symbology for the external scene when it is not visible,
hence enhancing situation awareness. The major disadvantages of HUDs are difficulties
in switching attention between the internal and external scene and difficulties
in detecting unexpected events. Despite this, there is currently no evidence
to suggest that the use of HUDs is associated with an increased risk of
accidents.
Withthe use ofHUDsincreasing throughoutcommercialaviation,moreresearch
is needed to identify ways to address the issues just noted. From a practical perspective,
one of the most pressing issues concerns training.Wedo not yet know whether
it is possible to train pilots to overcome the effects of cognitive tunneling, or how
muchtraining would be needed if itwaspossible to do so.Oneoption is to train pilots
to scan more effectively by teaching them to take their attention away from theHUD
and into the far domain (Wickens, Helleberg, Goh, Xu,\&Horrey, 2001). We could
16 CRAWFORD AND NEAL
Downloaded by [Jyvaskylan Yliopisto] at 07:02 22 December 2013
not find any studies in the public domain that have addressed this issue. Other issues
that need to be examined include the use of HUDs in multicrew operations, and the
effects of fatigue and experience on performance when using an HUD. Although
HUDsoffer significant benefits to airlines in both productivity and safety, an extensive
program of research is needed to ensure that these gains are realized. \citep{crawfordneal2006}

Results of this study suggest that automation bias is a significant factor in pilot
interaction with automated aids, and that pilots are not utilizing all available
information when performing tasks and making decisions in conjunction with
automation. Pilots exhibited the same overall rate of automation-related errors as
the student population in the Skitka et al. (1996) study, demonstrating that expertise
does not insulate individuals from automation bias. In fact, experience and expertise,
which might be predicted to make pilots more vigilant and less susceptible to
automation bias, were related to a greater tendency to use only automated cues. One
possible explanation for this may be that, because automated systems tend to be
highly reliable, more experience with them reinforces the notion that other cues are
merely redundant and unnecessary. Additionally, the higher-experienced pilots in
this study tended to be those more senior within their airline, and were currently
flying as captains. They may be used to delegating the cross-checking role to their
first officers rather than doing it themselves.
Although this study does not provide evidence that externally imposed accountability
affects the decision-making behavior of professional pilots, it does demonstrate
that the internalization of "accountability" for performance and strategies in
the use of automated systems impacts automation bias. Perceived accountability
was positively correlated with increased verification of automated functioning and
fewer omission errors. In other words, pilots who reported a higher internalized
sense of accountability for their interactions with automation verified correct
automation functioning more often and committed fewer errors than other pilots.
These results suggest that the sense that one is accountable for one's interaction
with automation encourages vigilance, proactive strategies, and the use of all
information in interactions with automated systems. The fact that the perception of accountability was not correspondent with our external manipulation indicates the
need to establish the degree to which accountability is a variable that can be
significantly influenced in pilots and other professional decision makers, who are
already functioning at a high level of personal responsibility for their conduct.
Alternatively, the lack of experimentally determined accountability effects could
be the artifact of a small sample, or the experimental manipulation could have been
confounded by a sense of accountability induced by the experience of participating
in a NASA study. The perception of accountability might also be part of some innate
cognitive style or personality construct, a hypothesis that will be addressed in a
future study.
Several aspects of the experimental tasks seemed to affect the tendency of
participants to verify the functioning of the automation; these included task importance,
predictability, and feedback. Descriptive data suggested that pilots were more
likely to catch automation events that involved altitude and heading than one that
involved a frequency error, corresponding with the typical ordering of flight
priorities as (a) aviate, (b) navigate, (c) communicate. The fact that the tracking
task automation failure so readily captured pilot attention can be explained in part
by the fact that the task display offered trend information (e.g., pilots could see
when the cursor was starting to drift out of the target area), as well as immediate
salient feedback on errors. In many real-world automation-aided tasks, pilots do
not receive either predictive information or immediate feedback on accuracy and
errors. Incorporating these features in the design of new automated systems may
be a way to facilitate vigilant use and aid in the detection of automation failures.
The finding that all pilots responded to the false engine fire event by shutting
down the indicated engine was unanticipated. Even more surprising was the fact
that their actions were contrary to responses on the debriefing questionnaires
indicating that some combination of cues would be necessary to diagnose "definitely
a fire," and that it would be safer, in the presence of only an EICAS message,
to retard the throttle of the indicated engine rather than shutting it down. Apparently,
these pilots were acting contrary to their self-described strategies. Part of the
explanation for their actions may rest in the "false memory7' data.
Pilots were definitely aware of the cues typically associated with an engine fire
and, in this time-pressured and cognitively demanding situation, were evidently
remembering a pattern of cues that was consistent with what should have been
present during the event. Real correlations among the cues--during an engine fire,
all of them should have been present--contributed to the illusion of their presence.
In similar situations during their careers (i.e., engine fire situations calling for the
shutdown of an engine), the pilots in this study would have been wrong if they had
not remembered the presence of several cues3 The more quickly they acted, the more "sure" they had to be that their actions were correct, and the more phantom
cues they remembered as having prompted their actions. Their judgments and their
memories were biased in the direction of confirming their expectations, and, as
Hamilton (1981) aptly remarked concerning correlated cues and expectations,
"[they] wouldn't have seen it if [they] hadn't believed it" (p. 137).
It is possible that this false memory phenomenon may be an additional mechanism
by which automation bias is bolstered. Pilots may not ever be aware that they
are using automated information as a short cut, either because the automated cues
are in fact consistent with other information (and no error results), or because errors
are not traced back to a failure to cross-check automated cues. In the false engine
fire event, for example, most pilots believed they had acted based on several cues,
and thus would not be prompted to change their strategies in the use of automated
information. Consistent with this analysis, pilots tended to falsely remember cues
that fit into their expectations but were not physically handled during the engine
shutdown. For example, they were less likely to remember the presence of an
illuminated fire handle (n = 2), which they had to manually press during engine
shutdown (and would have been forced to observe directly) than a master warning
light (n = 8), which was not directly manipulated as part of the shutdown procedure.
Explicit manipulation of a control made it more salient, and less likely to be
incorrectly remembered. \citep{mosierym1998}

CONCLUSIONS
In conclusion, this study has documented the existence of attentional tunneling as a
legitimate concern for the SV display when coupled with the HITS. Such concern certainly does not fully compromise the very real benefits of the HITS in supporting
low-workload trajectory guidance, and particularly of the substantial benefits of the
SV terrain representation for supporting terrain awareness. It does, however, point to
the importance of attentional training for pilots who use this new technology. \citep{wickens2009}

%Lähdeluettelo

\begin{thebibliography}{}

\bibitem[Bailey ym.(2007)Bailey, Randall E. , Kramer, Lynda J. \& Prinzel, Lawrence, III]{baileyym2007}
Bailey, Randall E. , Kramer, Lynda J. \& Prinzel, Lawrence, III 2007.
\textit{Fusion of Synthetic and Enhanced Vision for All-Weather Commercial Aviation Operations}.
NASA Technical Reports Server (NTRS), huhtikuu 2007.

\bibitem[Barrows \& Powell(1999)Barrows, A. \& Powell, D.]{barrowspowell1999}
Barrows, A. \& Powell, D. 1999.
\textit{Tunnel-in-the-sky cockpit display for complex remote sensing flight
trajectories}.
In Proceedings of the 4th International Airborne Remote Sensing Conference and 1st
Canadian Symposium on Remote Sensing, Ottawa, Canada: ERIM International, s.~I452--I459.

\bibitem[Beier \& Gemperlein(1994)Beier \& Gemperlein]{beiergemperlein2004}
Beier, Kurt \& Gemperlein, Hans 2004.
\textit{Simulation of Infrared Detection Range at Fog Conditions for Enhanced Vision Systems in Civil Aviation}.
Aerospace Science and Technology, 8, s.~63--71.

\bibitem[Bennet \& Flach(1994)Bennet \& Flach]{bennetflach1994}
Bennet, K.B. \& Flach, J.M. 1994.
\textit{When automation fails… }.
Human performance in automated systems: Current research and trends, s.~229--234.

\bibitem[Boeing(1996)Boeing]{boeing1996}
Boeing 1994.
\textit{Statistical summary of commercial jet aircraft accidents, worldwide operations, 1959–1995}.
Seattle, WA: Airplane Safety Engineering, Boeing Commercial Airplane Group.

\bibitem[Crawford \& Neal(2006)Crawford, Jennifer \& Neal, Andrew]{crawfordneal2006}
Crawford, Jennifer \& Neal, Andrew 2006.
\textit{A Review of the Perceptual and Cognitive Issues Associated With the Use of Head-Up Displays in Commercial Aviation}.
The International Journal of Aviation Psychology, 16:1, s.~1--19.

\bibitem[Etherington ym.(2000)Etherington, T.J., Vogl, T.L., Lapis, M.B., \& Razo, J.G.]{etherington2000}
Etherington, T.J., Vogl, T.L., Lapis, M.B., \& Razo, J.G. 2000.
\textit{Synthetic vision information system}.
Proceedings of the 19th Digital Avionics Systems Conference, s.~2.A.4--2.A.8.

\bibitem[Fadden ym.(2000)Fadden, S. ,Wickens, C.D., \& Ververs, P.M.]{faddenym2000}
Fadden, S. ,Wickens, C.D., \& Ververs, P.M. 2000.
\textit{Costs and benefits of head-up displays: An attention
perspective and a meta analysis (No. 2000-01-5542)}.
2000 World Aviation Congress, Warrendale, PA: Society of Automotive Engineers.

\bibitem[Fischer(1979)Fischer, E.]{fischer1979}
Fischer, E. 1979.
\textit{The role of cognitive switching in head-up displays}.
NASA Contractor Rep. No. 3137, Moffett Field, CA: NASA Ames Research Center.

\bibitem[Fischer ym.(1980)Fischer, E. , Haines, R.F. \& Price, T.A.]{fischerym1980}
Fischer, E. , Haines, R. F. \& Price, T. A. 1980.
\textit{Cognitive issues in head-up displays}.
NASA Tech. Paper No. 1711, Moffett Field, CA: NASA Ames Research Center.

\bibitem[Fiske \& Taylor(1994)Fiske, S.T. \& Taylor, S.E.]{fisketaylor1994}
Fiske, S. T. \& Taylor, S. E. 1994.
\textit{Social cognition (2nd Ed.)}.
New York: McGraw-Hill.

\bibitem[Foyle \& Hooey(2003)Foyle, D.C. \& Hooey, B.L.]{foylehooey2003}
Foyle, D.C. \& Hooey, B.L. 2003.
\textit{Improving evaluation and system design through the use of off-nominal testing: A methodology for scenario development}.
12th International Symposium on Aviation Psychology, Dayton, OH: Wright State University, s.~397--402.

\bibitem[Gerold(2001)Gerold, A.]{gerold2001}
Gerold, A. 2001.
\textit{Runway Incursions: The Threat on the Ground}.
Avionics Today.
Saatavilla WWW-muodossa
<http://www.aviationtoday.com/av/commercial/Runway-Incursions-The-Threat-on-the-Ground\_12628.html>. Viitattu 1.11.2014.

\bibitem[Hooey \& Foyle(2007)Hooey, B.L. \& Foyle, D.C.]{hooey2007}
Hooey, B.L. \& Foyle, D.C. 2007.
\textit{Aviation Safety Studies: Taxi Navigation Errors and Synthetic Vision Systems Operations}.
Human Performance Modeling in Aviation.

\bibitem[Horrey \& Wickens(2006)Horrey, W.J. , \& Wickens, C.D.]{horreywickens2006}
Horrey, W.J. , \& Wickens, C.D. 2006.
\textit{Examining the impact of cell phone conversations on driving using meta-analytic techniques}.
Human Factors, 48, s.~196-–205.

\bibitem[Korn ym.(2009)Korn, B. , Schmerwitz, S. , Lorenz, B. , \& Döhler, H.-U.]{kornym2009}
Korn, B. , Schmerwitz, S. , Lorenz, B. , \& Döhler, H.-U. 2009.
\textit{Combining enhanced and synthetic vision for autonomous all weather approach and landing}.
International Journal of Aviation Psychology, 19, s.~49-–75.

\bibitem[Larish \& Wickens(1991)Larish, I. , \& Wickens, C. D.]{larishwickens1991}
Larish, I. , \& Wickens, C. D. 1991.
\textit{Divided attention with superimposed und separated imagery: Implications for head-up displays}.
NASA Tech. Rep. No. 914, HUD 91:1, Savoy: University of Illinois, Aviation Research Laboratory.

\bibitem[Lauber ym.(1982)Lauber, J. K. , Bray, R. S. , Harrison, R. L. , Hemingway, J. C. \& Scott, B. C.]{lauberym1982}
Lauber, J. K. , Bray, R. S. , Harrison, R. L. , Hemingway, J. C. \& Scott, B. C. 1982.
\textit{An operational evaluation of head-up displays for civil transport operation: NASMFAA phase 111 final report}.
NASA Tech. Paper No. 1815, Moffett Field, CA: NASA Ames Research Center.

\bibitem[May \& Wickens(1995)May, P. A. \& Wickens, C. D.]{maywickens1995}
May, P. A. \& Wickens, C. D. 1995.
\textit{The role of visual attention in head-up displays: Design implications for varying symbol intensity}.
In Proceedings of the Human Factors and Ergonomics Society, 39. vuosikokous, s.~50--54. Santa Monica, CA: HFES.

\bibitem[Molloy \& Parasuraman(1996)Molloy, R. , \& Parasuraman, R.]{molloyparasuraman1996}
May, P. A. \& Wickens, C. D. 1995.
\textit{Monitoring an automated system for a single failure: Vigilance and task complexity effects}.
Human Factors, 38, s.~311--322.

\bibitem[Mosier ym.(1998)Mosier, Kathleen L. , Skitka, Linda J. , Heers, Susan, \& Burdick, Mark]{mosierym1998}
Mosier, Kathleen L. , Skitka, Linda J. , Heers, Susan, \& Burdick, Mark 1998.
\textit{Automation Bias: Decision Making and Performance in High-Tech Cockpits}.
The International Journal of Aviation Psychology, 8:1, s.~47--63.

\bibitem[Mosier ym.(1994)Mosier, K. L., Skitka, L. J. \& Korte, K.J.]{mosierym1994}
Mosier, K. L., Skitka, L. J. \& Korte, K.J. 1994.
\textit{Cognitive and social psychological issues in flight crew/automation interaction.}.
M. Mouloua \& R. Parasuraman (Eds.), Human performance in automated systems: Current research and trends. Hillsdale, NJ: Lawrence Erlbaum Associates, Inc. s.~191--197.

\bibitem[Möller \& Sachs(1994)Möller \& Sachs]{mollersachs1994}
Möller, H. \& Sachs, G. 1994.
\textit{Synthetic Vision for Enhancing Poor Visibility Flight Operations}.
IEEE AES Systems Magazine, maaliskuu 1994, s.~27--33.

\bibitem[Naish(1964)Naish, J. M.]{naish1964}
Naish, J. M. 1964.
\textit{Combination of information in superimposed visual fields}.
Nature, 202, s.~641--646.

\bibitem[Newman(1995)Newman, R. L.]{newman1995}
Newman, R. L. 1995.
\textit{Head-up displays: Designing the way ahead}.
Aldershot, England: Ashgate.

\bibitem[Newman(2000)Newman, R. L.]{newman2000}
Newman, R. L. 2000.
\textit{HUDs, HMDs, and SDO: A problem or a bad reputation (Report)}.
Wright-Patterson AFB, OH: Air Force Aerospace Medical Research Laboratory.

\bibitem[Nikolic ym.(2004)Nikolic, M. I. , Orr, J. M. , \& Sarter, N. B.]{nikolicym2004}
Nikolic, M. I. , Orr, J. M. , \& Sarter, N. B. 2004.
\textit{Why pilots miss the green box: How display context undermines attention capture}.
International Journal of Aviation Psychology, 14, s.~39--52.

\bibitem[Nordwall(1993)Nordwall, B.D.]{nordwall1993}
Nordwall, B.D. 1993.
\textit{HUD with IR System extends Pilot Vision}.
Aviation Week \& Space Technology, helmikuu 1993, s.~62--63.

\bibitem[Olmos ym.(2000)Olmos, O. ,Wickens, C.D., \& Chudy, A.]{olmosym2000}
Olmos, O. ,Wickens, C.D., \& Chudy, A. 2000.
\textit{Tactical displays for combat awareness: An examination of dimensionality and frame of reference concepts and the application of cognitive engineering}.
International Journal of Aviation Psychology, 10, s.~247--271.

\bibitem[Prinzel ym.(2013)Prinzel, Lawrence J. III, Arthur, Jarvis J. , Kramer, Lynda J. ,  Norman, Robert M. , Bailey, Randall E. , Jones, Denise R. ,  Karwac, Jerry R. Jr. , Shelton, Kevin J. \& Ellis, Kyle K. E.]{prinzel2013}
Prinzel, Lawrence J. III, Arthur, Jarvis J. , Kramer, Lynda J. ,  Norman, Robert M. , Bailey, Randall E. , Jones, Denise R. ,  Karwac, Jerry R. Jr. , Shelton, Kevin J. \& Ellis, Kyle K. E. 2013.
\textit{Flight-Deck Technologies to Enable NextGen Low Visibility Surface Operations}.
NASA Technical Reports Server (NTRS), toukokuu 2013.

\bibitem[Prinzel ym.(2004)Prinzel, Lawrence J. III, Comstock, J. Raymond Jr. , Glaab, Louis J. , Kramer, Lynda J. , Arthur, Jarvis J. \& Barry, John S.]{prinzel2004}
Prinzel, Lawrence J. III, Comstock, J. Raymond Jr. , Glaab, Louis J. , Kramer, Lynda J. , Arthur, Jarvis J. \& Barry, John S. 2004.
\textit{The Efficacy of Head-Down and Head-Up Synthetic Vision Display Concepts for Retro and Forward-Fit of Commercial Aircraft}.
The International Journal of Aviation Psychology, 14:1, s.~53--77.

\bibitem[Schnell ym.(2004)Schnell, Thomas,  Kwon, Yongjin, Merchant, Sohel \& Etherington, Timothy]{schnellym2004}
Schnell, Thomas,  Kwon, Yongjin, Merchant, Sohel \& Etherington, Timothy 2004.
\textit{Improved Flight Technical Performance in Flight Decks Equipped With Synthetic Vision Information System Displays}.
The International Journal of Aviation Psychology, 14:1, s.~79--102.

\bibitem[Stelzer \& Wickens(2006)Stelzer, E. M. , \& Wickens, C. D.]{stelzerwickens2006}
Stelzer, E. M. , \& Wickens, C. D. 2006.
\textit{Pilots strategically compensate for display enlargements in surveillance and flight control tasks}.
Human Factors, 48, s.~166--181.

\bibitem[Strayer \& Drews(2007)Strayer, D.L. \& Drews, F.A.]{strayerdrews2007}
Strayer, D.L. \& Drews, F.A. 2007.
\textit{Multitasking in the automobile}.
A.F. Kramer, D.A. Wiegmann, \& A. Kirlik (Eds.), Attention: From theory to practice, Oxford, UK: Oxford University Press, s.~121--133.

\bibitem[Strayer ym.(2001)Strayer, D.L. , Drews, F.A. \& Johnston, W.A.]{strayerym2001}
Strayer, D.L. , Drews, F.A. \& Johnston, W.A. 2001.
\textit{Driven to distraction: Dual-task studies of simulated
driving and conversing on cellular telephone}.
Psychological Science, 12, s.~462-–466.

\bibitem[Ververs \& Wickens(1996)Ververs, Patricia May \& Wickens, Christopher D.]{ververswickens1996}
Ververs, Patricia May \& Wickens, Christopher D. 1996.
\textit{Allocation of attention with head-up displays}.
Tech. Rep. ARL–96–1/FAA–96–1, Savoy: University of Illinois Institute of Aviation.

\bibitem[Ververs \& Wickens(1998)Ververs, Patricia May \& Wickens, Christopher D.]{ververswickens1998}
Ververs, Patricia May \& Wickens, Christopher D. 1998.
\textit{Head-Up Displays: Effect of Clutter, Display Intensity, and Display Location on Pilot Performance}.
The International Journal of Aviation Psychology, 8:4, s.~377--403.

\bibitem[Weintraub \& Ensing(1992)Weintraub, D. J. \& Ensing, M.]{weintraubensing1992}
Weintraub, D. J. \& Ensing, M. 1992.
\textit{Human factors issues in head-up display design: The book of
HUD}.
Wright-Patterson Air Force Base, OH: Crew Station Ergonomics
Information Analysis Center., (SOAR, CSERIAC 92–2).

\bibitem[Wickens \& Alexander(2009)Wickens, Christopher D. \& Alexander, Amy L.]{wickensalexander2009}
Wickens, Christopher D. \& Alexander, Amy L. 2009.
\textit{Attentional Tunneling and Task Management in Synthetic Vision Displays}.
The International Journal of Aviation Psychology, 19:2, s.~182--199.

\bibitem[Wickens \& Andre(1990)Wickens, C.D. \& Andre, A.D.]{wickensandre1990}
Wickens, C.D. \& Andre, A.D. 1990.
\textit{Proximity compatibility principle and information display: Effects
of color, space, and objectness on information integration}.
Human Factors, 32, s.~61--77.

\bibitem[Wickens \& Hollands(2000)Wickens, C.D. \& Hollands,  J.G.]{wickenshollands2000}
Wickens, C.D. \& Hollands, J.G. 2000.
\textit{Engineering psychology and human performance (3. painos)}.
Upper Saddle River, NJ: Prentice-Hall.

\bibitem[Wickens \& Long(1995)Wickens, C.D. \& Long, J.]{wickenslong1995}
Wickens, C.D. \& Long, J. 1995.
\textit{Object vs. space-based models of visual attention: Implication for the design of head-up displays}.
Journal of Experimental Psychology: Applied, 1, s.~179--194.

\bibitem[Wickens ym.(2003)Wickens, C.D. , Muthard, E.K. , Alexander, A.L. , Van Olffen, P. \& Podczerwinski, E.]{wickensym2003}
Wickens, C.D. , Muthard, E.K. , Alexander, A.L. , Van Olffen, P. \& Podczerwinski, E. 2003.
\textit{The influences of display highlighting and size and event eccentricity for aviation surveillance}.
Proceedings of the 47th annual meeting of the Human Factors \& Ergonomics Society, Santa Monica, CA: HFES.

\bibitem[Williams ym.(2001)Williams, D. , Waller, M. , Koelling, J. , Burdette, D. , Doyle, T. , Capron,W. , ym.]{williamsym2001}
Williams, D. , Waller, M. , Koelling, J. , Burdette, D. , Doyle, T. , Capron,W. , ym. 2001.
\textit{Concept of operations for commercial and business aircraft synthetic vision systems}.
NASA Tech. Memo. No. TM-2001-211058.

\bibitem[Zuschlag(2003)Zuschlag, M.]{zuschlag2003}
Zuschlag, M. 2003.
\textit{Certifying head-up displays}.
Human Factors Newsletter. 02--22.

\end{thebibliography}

\end{document}
