% !TeX encoding = utf8
%
% [ Tiedostossa käytetty merkistö on utf8, vaihtoehtoisesti voisi olla esim.]
% [ ISO 8859-1 eli Latin 1. Ylläoleva rivi ]
% [ tarvitaan, jos käyttää MiKTeX-paketin mukana tulevaa TeXworks-editoria. ]
%
% TIETOTEKNIIKAN KANDIDAATINTUTKIELMA
%
% Yksinkertainen LaTeX2e-mallipohja kandidaatintutkielmalle.
% Käyttää Antti-Juhani Kaijanahon kirjoittamaa gradu3-dokumenttiluokkaa.
%
% Jos kirjoitat pro gradu -tutkielmaa, tee mallipohjaan seuraavat muutokset:
%  - Poista dokumenttiluokasta optio bachelor .
%  - Poista makro \type .
%  - Lisää suuntautumisvaihtoehto makrolla \studyline .
%  - Lisää tieto ohjaajasta makrolla \supervisor .

\documentclass[utf8,bachelor,manualbib]{gradu3}

\usepackage{palatino} % valitaan oletusfonttia hieman tyylikkäämpi fontti

\usepackage{graphicx} % tarvitaan vain, jos halutaan mukaan kuvia
\usepackage{amsmath}  % tarvitaan käytettäessä monimutkaisten matemaattisten kaavojen ja \eqref-kaavaviittauksen yhteydessä
\usepackage{url} % tarvitaan \url-komentoa varten
\usepackage{booktabs}

% Otetaan käyttöön author-date-järjestelmän mukaiset lähdeviittaukset:
\usepackage{natbib}
% Vaihdetaan kirjoittajan nimen ja vuosiluvun väliseksi erottimeksi
% välilyönti (oletuserottimena on pilkku):
%\bibpunct{(}{)}{;}{a}{}{,}


% HUOM! Tämän tulee olla viimeinen \usepackage koko dokumentissa!
\usepackage[bookmarksopen,bookmarksnumbered,linktocpage]{hyperref}

%\addbibresource{viite.bib}% Lähdetietokannan tiedostonimi
%http://www.tex.ac.uk/tex-archive/macros/latex/exptl/biblatex-contrib/biblatex-chicago/latex/biblatex-chicago.sty
%http://www.tex.ac.uk/tex-archive/macros/latex/contrib/etoolbox/etoolbox.sty
%http://mirrors.med.harvard.edu/ctan/macros/latex/contrib/biblatex/latex/biblatex.sty
%http://ctan.mackichan.com/macros/latex/contrib/biblatex/latex/biblatex2.sty
%http://mirror.hmc.edu/ctan/macros/latex/contrib/logreq/logreq.sty
%https://github.com/Martin-Rotter/qt-survival-guide/blob/master/logreq.def

\begin{document}

\title{SV- ja EV-järjestelmät kaupallisessa ilmailussa}

\translatedtitle{Synthetic and Enhanced Vision Systems in Commercial Aviation}

%\studyline{}
\avainsanat{SVS, EVS, ilmailu, näkyvyys, HUD, HDD, lentäminen, lentoturvallisuus, IFR, NextGen, NGATS, EVO}
\keywords{SVS, EVS, synthetic vision, enhanced vision, aviation, HUD, HDD, flying, safety, visibility, IFR, NextGen, NGATS, EVO}
\tiivistelma{Tähän tulee tiivistelmä (tausta, tavoite, tulokset, johtopäätökset).
}
\abstract{Tähän tulee englanninkielinen versio tiivistelmästä.
}

\author{Matias Laitinen}
\contactinformation{\texttt{matias.laitinen@gmail.com}}
% jos useita tekijöitä, anna useampi \author-komento
%\supervisor{TODO}
% jos useita ohjaajia, anna useampi \supervisor-komento
%\type{bachelor} % tämän makron oletus on "pro gradu -tutkielma" ja bachelor-optiolla kandidaatintutkielma

\maketitle
  
\mainmatter

\chapter{Johdanto}

Viime aikoina lentoturvallisuus on noussut useasti esille medioissa, kun sekä harraste- että kaupallisen ilmailun puolella on tapahtunut monenlaisia lento-onnettomuuksia tai ilmailun vaaratilanteita. Nämä onnettomuudet aiheutuvat useimmiten inhimillisistä virheistä. Tutkimuksen tavoitteena on ottaa selvää, millä eri tavoin olisi keinotekoisia näköjärjestelmiä käyttämällä mahdollista ehkäistä lento-onnettomuuksia ja parantaa lentäjän tilannetietoisuutta, etenkin huonon näkyvyyden olosuhteissa. Tällaiset järjestelmät ovat olleet sotilaspuolen käytössä jo pitkän aikaa, mutta siviili-ilmailussa niitä hyödynnetään vasta melko vähän. Kartoittamalla näiden järjestelmien kustannuksia ja käytettävyyttä saadaan toivottavasti tehtyä jonkinlaisia johtopäätöksiä niiden soveltuvuudesta käytäntöön. TODO: Kerro paremmin ja laita viitteet kuntoon.

\emph{TODO:}
The United States air transportation system is undergoing a transformation to accommodate a projected 3-fold
increase in air operations by 2025.1 Technological and systemic changes are being developed to significantly
increase the capacity, safety, efficiency, and security for this Next Generation Air Transportation System
(NGATS). One of the key capabilities envisioned to achieve these goals is the concept of Equivalent Visual
Operations (EVO), whereby Visual Flight Rules (VFR) operational tempos and also, perhaps, operating
procedures (such as separation assurance) are maintained independent of the actual weather conditions. One
methodology by which the goal EVO might be attainable is to create a virtual visual flight environment for
the flight crew, independent of the actual outside weather and visibility conditions, through application of
Enhanced Vision (EV) and Synthetic Vision (SV) technologies. \citep{baileyym2007}

\emph{TODO:}
The NASA SVS project is based on the premise that better pilot SA during
low-visibility conditions can be achieved by reducing the steps required to build a
mental model from disparate pieces of data through the presentation of how the
outside world would look to pilots if their visibility was not restricted. \citep{prinzel2004}

\chapter{Ilmailu ja näkyvyys}

\section{Näkyvyyden vaikutus lennon aikana}

Näkyvyys on lentokoneen ohjaajalle tärkeää lennettäessä lähellä maata, etenkin laskeutumisen aikana. Sen vuoksi huonot näkyvyysolosuhteet aiheuttavat suuria rajoitteita lentotoiminnalle \citep{mollersachs1994}. Ohjaajan lentonäkyvyyteen vaikuttavat monet meteorologiset olosuhteet, kuten pimeys, pöly, sumu ja sade \citep{wickens2009}. Erityisesti sumuisissa olosuhteissa näkyvyys voi huonontua voimakkaasti ja ulkomaailman yksityiskohtia on miltei mahdotonta erottaa \citep{beiergemperlein2004}.

Kaikilla lentokentillä toimittaessa ovat voimassa tietyt näkyvyysrajoitukset. Kentillä, joilla on käytössä esimerkiksi ILS:n kaltaisia lähestymisapuja, on tietty minimi, josta tulee olla mahdollista jatkaa lähestymistä visuaalisesti. Vakavimmillaan rajoitukset vaikuttavat kentillä, joilla ei tällaisia järjestelmiä ole käytössä. Nämä säännöt ovat voimassa, vaikka lähestyvällä koneella olisikin käytössään nykyaikaiset mittari- ja navigointilaitteet. \cite{mollersachs1994}.

Jo ilmailun alkuajoista asti on ilmailuteollisuus kehittänyt laitteita, joilla voittaa näitä huonon näkyvyyden rajoituksia. Tällaisia voivat olla esimerkiksi lentoasentojärjestelmät, navigointilaitteet, mittarilähestymislaitteet (Instrument Landing System, ILS), liikkuvat karttalaitteet sekä korkeasta maastosta varoittavat järjestelmät (Terrain Awareness and Warning System, TAWS). Näissä voidaan kuitenkin havaita se ongelma, että kaikki nykyaikaisetkin informaation esittämiseen tarkoitetut järjestelmät vaativat ohjaajilta jatkuvaa tiedonhakua ja -käsittelyä pysyäkseen selvillä ilma-aluksensa tulevista liikkeistä huonon näkyvyyden olosuhteissa \citep {prinzel2004}.

FSFN:n(Flight Safety Foundation) mukaan miltei 60 \% kaupalliseen lentotoimintaan liittyvistä maahansyöksyonnettomuuksista on tapahtunut lähestymisen tai laskeutumisen aikana. Etheringtonin ym. \citeyearpar{etherington2000} mukaan ohjattavissa olevan ilma-aluksen törmäys maastoon (Controller Flight Into Terrain, CFIT) on vallitseva onnettomuustyyppi ja vastuussa yli puolesta kaupallisen ilmailun onnettomuuksista. Myös Lentokonevalmistaja Boeingin \citeyearpar{boeing1996} tilastot tukevat tätä näkemystä. CFIT- onnettomuudet liittyvät yleensä paikka- tai asentotiedon (Situational Awareness, SA) menetykseen lähestymis- ja laskuvaiheessa, kun ohjaajat menettävät käsityksensä suunnasta, korkeudesta ja suhteesta ympäristöön \cite{schnellym2004}.

\section{Näkyvyyden vaikutus maatoimintaan}

Prinzel ym. \citeyearpar{prinzel2013} kertovat huonon näkyvyyden aiheuttamien toiminnan hidastumisten ja viivästysten maatoiminnassa olevan kasvavasti vaikuttamassa myös ilmatilankäytön viiveisiin. Huonon näkyvyyden olosuhteissa ohjaajien ja ajoneuvonkuljettajien tulee säilyttää tilannetietoisuutensa varmistaakseen, että maatoiminta on turvallista ja tehokasta.

Esimerkiksi FAA:n vuoden 2010 turvallisuusselvityksen mukaan 52 928 316 maatoimintaan liittyvän tapahtuman aikana tapahtui 951 kiitotiepoikkeamaa (Runway Incursion), joista 12 oli vakavia. Vaikka tämä luku on suhteessa pieni, kiitotiepoikkeamalla voi olla tuhoisat seuraukset. Suurimpana syynä näissä tapauksissa oli ohjaajan inhimillinen erehdys (63 \%). Tilannetietoisuutta parantamalla voitaisiin siis saada merkittävästi vähennettyä kiitotiepoikkeamien määrää. \citep{prinzel2013}.

Tilannetietoisuutta maatoiminnassa voitaisiin parantaa esimerkiksi käyttämällä infrapunakameroita jopa tiheän sumun olosuhteissa, jotta havaittaisiin paremmin esteitä, kuten henkilöstöä, ajoneuvoja ja laitteistoja. \citep{beiergemperlein2004}.













\chapter{SV- ja EV-järjestelmät}

\section{Synthetic Vision}

Keinonäköjärjestelmillä (Synthetic Vision System, SVS, Synthetic Vision Information System, SVIS) tarkoitetaan keinotekoisen ympäristökuvan luomista tietokoneella \citep{baileyym2007} tai NASA:n Rockwell Collinsin kanssa kehittämää tällaista teknologiaa käyttävää järjestelmää, SVS \citep{crawford2006}. Prinzelin ym. \citeyearpar{prinzel2004} mukaan NASA:n ilmailuturvallisuusohjelman osana SVS:n tarkoituksena on eliminoida huono näkyvyys lento-onnettomuuksien aiheuttajana sekä parantaa yleis- liike- ja kaupallisen ilmailun operationaalisia valmiuksia.

Kuva luodaan yhdistäen lentoasento- ja tarkkuusnavigointijärjestelmiltä sekä maasto- ja estetietokannoista saatua lennon kannalta tärkeää tietoa. \cite{schnellym2004} kertovat SVIS -järjestelmien olevan uuden sukupolven ohjaamojärjestelmä, joka tulee olemaan tärkeässä osassa tulevaisuuden kaupallisten lentokoneiden ohjaamoissa.








\emph{TODO:}
SVS is a different flight display system developed by Rockwell Collins and
NASA. It provides a conformal view of the world outside using the EGPWS database
to portray a picture of terrain, obstructions, and airports. SVS uses highway-
in-the-sky software, giving pilots visual cues and flight path guidance. As the
images provided are database derived, it is possible that important information is
not available when needed, such as a new construction. Furthermore, processing
latency or database integrity issues may lead to confusion when the pilot breaks
through cloud cover and the image from the HUD is not the same as that provided
by the SVS.\citep{crawford2006}



\section{Enhanced Vision}

Parannetun näön järjestelmillä (Enhanced Vision System, EVS tai Enhanced Flight Vision System, EFVS) tarkoitetaan elektronisen apuvälineen, kuten lämpökameran (Forward-Looking Infrared, FLIR) tai millimetritutkan (Millimeter Wave Radar, MMWR) avulla näytettyä kuvaa ulkomaailmasta \citep{baileyym2007}. Möller ja Sachs \citeyearpar{mollersachs1994} kertovat optisten järjestelmien olevan passiivisia laitteita, joilla voidaan muodostaa ympäristökuva ilman tietoa etäisyyksistä. Tutkalla sen sijaan saadaan aktiivisesti etäisyystietoa ympäristöstä, mutta tavallisen näköistä kuvaa on vaikea muodostaa.

\emph{TODO:}
EVSs are designed to increase safety in low-visibility conditions. EVSs can be
added to both HUDs and HDDs, providing the pilot with infrared-derived visual
cues of the external scene, including terrain and traffic. The pilot sees the
real-world runway even when it is obscured by poor weather conditions. The
EVS also enables views of surrounding terrain in poor lighting and weather conditions.
This may help to prevent controlled flight into terrain.
The use of EVS is limited during periods of heavy rain, fog, and dust. Furthermore,
as infrared only shows thermal differences, the images can be confusing during
certain conditions such as when objects within a scene are of an equal
temperature.\citep{crawford2006}

\section{Heijastusnäytöt} \emph{TODO:}

SVS/EVS-tietoa voidaan esittää esimerkiksi heijastusnäyttöjen (Head-Up Display, HUD) avulla.

An HUD is a projected display of symbology on a transparent screen. As can be
seen in Figure 1, the symbology is superimposed over the pilot’s forward field
of view, enabling him or her to monitor primary flight information while maintaining
a view of the outside world. The result is an opportunity for less time
“head down” looking at instrumentation approximately 10 degrees below the line of
sight, and more time maintaining visual contact with the outside environment.
The first use of anHUDwas for gun sights in the 1950s. These early HUDs were
used for aiming and not as a flight instrument. In 1960, the Hawker Siddeley Buccaneer
included the first operational HUD designed for use as a piloting instrument
(Weintraub\&Ensing, 1992).

ThisHUDconsisted of a horizon and a reference symbol
of an aircraft. Altitude and speed values were digitally displayed, with the Flight
Director providing rough flight path guidance information. TheHUDsymbology of
the Buccaneer provides the basis for that used in most HUDs today.
HUDs are currently used as a visual aid to assist during two main landing situations,
namely the visual approach and the transition from instrument meteorological
conditions to a visual landing. The use of HUD during the various phases of
flight is mandated by the airline company operating the aircraft in which the HUD
is installed. The four advantages for the incorporation of HUDs into modern aircraft
that have been proposed are a reduction of head-down time during critical
stages of flight, a potential reduction in the need to refocus the eyes from the near
to the far domain (from instrumentation to the external world), an improvement in
awareness of the external domain, and an improvement in the quality of the instru-mentation display by comparison to conventional displays based on dials and
gauges (May \& Wickens, 1995).

]Head-up displays (HUDs) are being introduced into general aviation in part due to
the advantages shown in both military and civilian aviation. There are general
findings of the success of HUDs when these are compared with either the typical
instrumentation panel or the same symbology presented in the head-down position.
llUDs have continually revealed performance benefits for flight-path maintenance
(Fischer, Haines, \& Price, 1980; Lauber, Bray, Harrison, Hemingway, \& Scott, 1982; Wickens \& Long, 1995) and detection of expected incidents or warnings
(e.g., detecting the runway on approach) occurring either in the scene or on the
symbology itself (Fischer, 1979; Larish \& Wickens, 1991; Wickens \&Long, 1995).

Studies of flight performance have shown advantages for HUDs over traditional
head-down displays (HDDs), including superior flight path maintenance
and higher precision landings (see Fischer, Haines, \& Price, 1980; Naish, 1964).
Furthermore, for some airports and aircraft types, HUDs enable lower visibility
takeoffs and landings than previously possible. This can provide significant cost
savings for airlines. However, potential problems have also emerged. Some of the
early problems relating to design appear to have been resolved, but there are a
number of cognitive issues that are still a matter of debate and research. These include
the effects of divided attention and cognitive tunneling, and spatial disorientation
and unusual attitude recovery. These issues are reviewed in later sections. \citep{crawford2006}

\section{Lentotoiminnassa saavutettavia etuja}

Bennet ja Flach \citeyearpar{bennetflach1994} väittävät tiedon näyttämisen ohjaamoissa dynaamisten ja graafisten esitysten avulla johtavan ihmiskeskeisempään malliin, jossa jatkuva tiedonsaanti näyttäisi luonnollisen kaltaiset rajat turvalliselle toiminnalle. Tällöin korostuisi ihmisen joustavuus käyttää luonnollista ja koodattua informaatiota parhaiten hyväkseen.

Luonnollisella informaatiolla käsitetään tietoa, jota saadaan samalla tavoin kuin näkölento-olosuhteissa katsomalla ulos ohjaamosta. Koodattu informaatio sen sijaan vaatii ohjaajalta erikseen sen todellisen arvon ymmärtämistä. Luonnollista informaatiota voi olla esimerkiksi korkeuden silmämääräinen arviointi ja koodattua informaatiota korkeusmittarin näyttämä. \citep{prinzel2004}. SV -järjestelmien avulla voidaan esittää tällaista luonnollisen kaltaista ja intuitiivista informaatiota, jota on helppo käsitellä \citep{wickensandre1990}.

SVS-teknologia voi mahdollistaa rajoitetusta näkyvyydestä aiheutuvien ongelmien ratkaisemisen visuaalisesti, tehden lentotoiminnasta säästä riippumatta samanlaista kuin kirkkaassa päivänvalossa ja parantaen ohjaajien tilannetietoisuutta \citep{prinzel2004}. Koska rajoittunut näkyvyys on suurimpana tekijänä monissa kohtalokkaissa lento-onnettomuuksissa \citep{boeing1996}, voisi SVS:n käytöllä olla merkittävä vaikutus turvallisuuteen. Jo pelkät lentoturvallisuuden hyödyt, joita SVS mahdollistaa, ovat tarpeeksi aiheen tutkimiselle, mutta koska kyseinen järjestelmä on hyvin kallis, on löydettävä myös operationaalisia ja taloudellisia hyötyjä \citep{prinzel2004}.

Näitä hyötyjä ainakin NASA \citeyearpar{williamsym2001} arvioi voitavan saavuttaa kasvavan lentoliikenteen myötä esimerkiksi seuraavien etujen kautta:

\begin{itemize}
\item Pienempi ajankäytön tarve kiitotiellä huonon näkyvyyden olosuhteissa
\item Pienemmät lähtö - ja tulominimit
\item Helpommin sallittavat erilaiset lähestymistavat, etenkin rinnakkaisille kiitoteille
\item Pienemmät porrastukset saapuvien ilma-alusten välillä
\item Toisistaan riippumattoman toiminnan mahdollistaminen rinnakkaisilla, lähekkäin sijaitsevilla kiitoteillä
\end{itemize} 

Myös Schnellin ym. \citeyearpar{schnellym2004} mukaan SVIS-järjestelmät antavat ohjaajille tehtäväkohtaista tietoa ja opastusta, jota tarvitaan lennettäessä monimutkaisia, kaartuvia lähestymispolkuja. Lisäksi hekin mainitsevat nykyisen ilmatilan olevan varsinkin joillain lähestymisalueilla kapasiteettinsa rajoilla. Jo pieniki muutos säätilassa tai laitteisto-ongelmat lentokentällä voi aiheuttaa liikenteen ruuhkautumisen. SVIS-järjestelmistä kaavaillaan mahdollista ratkaisua tähän. 

\cite{baileyym2007} väittävät SV-järjestelmillä saatavan huomattavia parannuksia maastoestetietouteen ja että se voisi vähentää CFIT-onnettomuuksien riskiä nykyiseen ohjaamoissa käytettävään teknologiaan verrattuna. Myös \cite{schnellym2004} toteavat, että SVIS voisi olla avainteknologia CFIT-onnettomuuksien vähentämisessä.



\emph{TODO:}
Researchers are currently exploring the possibility ofcombiningEVSandSVSto
produce an enhanced synthetic vision system. With both systems used in combination,
the images, which will include possible hazardous traffic, will give the pilot the
sameinformation aswouldbe available for a daylight, clear weather visual flight.As
both systems can be presented on anHUDtogether, human factor design principles,
pilot performance, problems, and safety concerns will need to be investigated. \citep{crawford2006}


Nordwallin \citeyearpar{nordwall1993} mukaan heijastusnäyttö (Head-Up Display, HUD)-lämpökamera -yhdistelmällä saavutetaan esimerkiksi sumussa huomattavasti parempi kuva ynpäristöstä, kuin mitä paljaalla silmällä voitaisiin saavuttaa. Beierin ja Gemperleinin \citeyearpar{beiergemperlein2004} tutkimuksen perusteella aivan tiheässä sumussa, jossa näkyvyydet ovat 300m, 100m tai vähemmän, ei lämpökameroita käyttämällä kuitenkaan saada parannusta näkyvyyteen. Tässä tapauksessa lähestymisessä tarvittaisiin esimerkiksi tutkajärjestelmä ympäristökuvaa luomaan.

SV-EV yhdistäminen, tähän kuva \citep{mollersachs1994}

\emph{TODO:}
2.1 Integrated and Fused Synthetic and Enhanced Vision Systems Concepts
The complementary capabilities of SV and EV have been well-recognized6 with the premise that7 “the
strengths of enhanced system can compensate for the deficiencies in the synthetic system and that the
strengths of synthetic system can compensate for the deficiencies in the enhanced vision system.” While
these goals are obvious, optimal methods and capabilities are not on hand.
2.2 Previous Research
Several studies8-10 have shown that the optimal combination of SV and EV technology likely uses the direct
display of SV to the flight crew without direct display of EV, but instead, using EV “behind the glass” for
navigation error detection, database integrity monitoring, and real-time obstacle/object detection. Image
processing performs these functions automatically without intervention by the flight crew. This arrangement
provides a highly usable display presentation (i.e., SV) that is impervious to the actual weather and visibility
conditions, yet if un-charted obstacles, database errors or navigation errors are detected by the EV running in
the background, the situation is annunciated, and almost “perfect” decision-making by the pilot occurs.8-10 In
fact, the study conclusion from Parrish8 – “SVS concepts should not be implemented without incorporating
image-processing decision aides for the pilot” – launched a 5 year effort at NASA and elsewhere developing
SV Systems (SVS) enabling technologies for database integrity monitoring and EV object detection. 11-15
While degrees of success in developing these decision aids have been met, technology for “perfect” object
detection and database/navigation error detection does not yet exist. Further, there may always be gaps which
may still warrant flight deck procedures and human interventions for integrity and error checks.
\citep{baileyym2007}

\emph{TODO:}
Present federal airways, on en route charts, are depicted as primarily straight
lines between ground-based navigation stations (VORs). To go from Point A to
Point B, a flight crew is generally expected to file a flight plan following federal
airways (Victor airways or Juliet airways). The result is a routing system that produces
flight plans that look more like driving on roads and highways than flying in
airspace. Quite often Points A and B are not neatly connected by a straight line, but
rather, the crew has to make several turns and change airways at intersections. The
present airway system, however, has some advantages. Following an airway at or
above the minimum enroute altitude ensures terrain clearance and adequate radio
reception of the navaids. Pilots can request direct flight paths if they ensure adequate
clearance with terrain. SVIS is a concept that may help crews to increase
their terrain awareness when operating en route and in terminal areas.
Most present approaches involve straight segments between waypoints. However,
modern technology allows for the execution of curved approaches in four dimensions
(three spatial dimensions and the time dimension). Navigation technology
based on global positioning satellites, combined with SVIS, will enable the
design of curved and constant descent approaches that may be more efficient than
navaid-based, straight-line approaches with step-down fixes. Although technically
feasible, such complex curved paths cannot easily be communicated to and
managed by flight crews with conventional primary flight displays (PFDs) and
multifunction displays (MFDs). Curved approaches, especially when including
the time dimension, cannot easily be conveyed in verbal or paper format.
The SVIS displays evaluated in this study allow communication of the desired
curved flight path by means of a tunnel (corridor) that the pilot needs to follow.
SVIS is based on the concept of a tunnel in the sky depicting a three-dimensional
view of the world and the desired flight path (Barrows \& Powell, 1999). \citep{schnellym2004}

Ehkä kuvia tähän vielä \citep{schnellym2004} \emph{TODO:}



\emph{TODO:}
NASA is conducting research, development, test, and evaluation of flight deck display technologies that may significantly enhance the flight crew's situation awareness, enable new operating concepts, and reduce the potential for incidents/accidents for terminal area and surface operations. The technologies that form the backbone of the BTV operational concept include: surface and airport moving maps; head-up and head-worn displays; four dimensional trajectory (4DT) guidance algorithms; digital data-link communications; synthetic and enhanced vision technologies; and traffic conflict detection and alerting systems (Bailey, Prinzel, Young, and Kramer, 2011; Prinzel et al., 2011). Preliminary research is described assessing a subset of these technologies in comparison to current-day low visibility surface operations. \citep{prinzel2013}











































\section{Maatoiminnassa saavutettavia etuja}

Maatoiminnassa lennonjohdon, koneiden ohjaajien sekä ajoneuvonkuljettajien tilannetietoisuutta pyritään pitämään yllä tarjoamalla visuaalisia merkkejä omasta sijainnista, kulkureiteistä ja tilasta kiito- ja rullausteillä, odotuspaikoilla ja asematasoilla. Tämä hoidetaan esimerkiksi valojen, merkintöjen ja opasteiden avulla. Tällaisia järjestelmiä kutsutaan nimellä Surface Movement Guidance and Control System (SMGCS). \citep{prinzel2013}.

Tilannetietoa ylläpitäviä järjestelmiä voitaisiin myös käyttää ohjaamoissa. Tällaisista järjestelmistä voisi Prinzelin ym. \citeyearpar{prinzel2013} mukaan olla hyötyä, varsinkin henkilöstön näkyvyyden parantamisessa keinotekoisesti sekä tilannetietoisuuden (paikka- ja reittitiedon ja mahdollisesti myös liikenne- ja estetiedon) parantamisessa erilaisten karttajärjestelmien avulla. Etenkin yöllä, tai savun tai pölyn haitatessa näkyvyyttä EV-järjestelmät voivat auttaa pilottia toimimaan turvallisemmin maassa \citep{prinzel2013}. Hooeyn \& Foylen \citeyearpar{hooey2007} tutkimuksen mukaan 17\% yö- tai huonon näkyvyyden olosuhteissa tapahtuneessa rullauskokeessa tuli esille navigointivirheitä, jotka saatiin korjattua liikkuvien lentokenttäkarttojen (Airport Moving Map, AMM) avulla.

\emph{TODO:}
Synthetic Vision information provides also a menas for enhancing poor visibility operations of aircraft when moving on the ground. There are only simple or marginal optical aids for surface movement.
For example, such aids concern lighting of center lines of
taxiways, stop and clearance bars or signs.
With use of synthetic vision, the information of the pilot for
controlling the aircraft on the ground can be significantly
enhanced. This goal may be achieved by providing guidance
information based on a map representing the airport area \citep{mollersachs1994} 

\emph{TODO:}
The results demonstrated that an enhanced flight vision system may potentially enhance situation awareness and ameliorate problems witnessed when visibility drops requiring the use of LVO/SMGCS enhanced visual aids. However, the use of EV alone was not found to substantially enhance surface operations compared to baseline (i.e., no FLIR) without the addition of an AMM. Pilots consistently rated the AMM to be of significant value for these operations and, together, the EV and AMM was rated to be of tremendous benefit in maintaining SA and workload during 300 RVR approach and departures with simulated taxi-in and -out. The results also fully support the potential direction that EV with an AMM may provide an “operational credit” for SMGCS wherein an operator, with these requisite flight deck technologies, may be able to conduct lower than 500 RVR operations at airports that may only have a Level 1 LVO/SMGCS airport visual aids in place. Another option may be to enable under 1200 RVR surface operations at airports that do not have any LVO/SMGCS airport visual aids in place.
The FAA has stated that, “taxiing on the airport surface is the most hazardous phase of flight” (Gerold, 2001). Almost a decade later, that statement still rings true, but LVO/SMGCS enhanced visual aids and other controls are significantly improving this situation. Emerging flight deck technologies offer a potential means to create an equivalent level of safety and performance. These flight deck technologies, such as the E-SMGCS -AMM display and EV, could assist in fully realizing the potential of NextGen by offering a more affordable path toward safe and efficient LVO/SMGCS operations through an “equivalent visual” paradigm.\citep{prinzel2013}

\emph{TODO:}
The use of AMMs to enhance situation awareness has long been established. However, little research exists that
investigated their use under low visibility surface operations. To date, research has been limited to visibility
conditions greater than 700 RVR and without the enhanced visual aids required under LVO/SMGCS operations.
In this research, the AMM design was based on NASA “best practices” for AMMs, existing published standards,
and an AMM information priority survey. The survey was collected on twenty commercial pilots, experienced with
SMGCS and AMMs, based on established methods (Schveneveldt, et al., 2000; Yeh \& Chandra, 2003). The results
evinced that pilots desired the following SMGCS elements be depicted under low visibility surface operations: Geographic Position Markers (GPMs), clear route, stop bars, and hold lines. These additional informational elements are added to the AMM to create an E-SMGCS display concept.
The E-SMGCS concept is a display mode of the AMM that is invoked by the flight crews when known LVO/SMGCS conditions exists (i.e., under 1200 RVR). The AMM then shows specific LVO/SMGCS information elements based on information priority survey. The E-SMGCS mode retains all normal AMM functionality and also provides the pilots with specific symbology to enable the pilots to cognitively map task priority elements between the AMM, paper charts, and out-the-window visual cues.

\emph{TODO:}
Conclusion
The results demonstrated that an enhanced flight vision system may potentially enhance situation awareness and ameliorate problems witnessed when visibility drops requiring the use of LVO/SMGCS enhanced visual aids. However, the use of EV alone was not found to substantially enhance surface operations compared to baseline (i.e., no FLIR) without the addition of an AMM. Pilots consistently rated the AMM to be of significant value for these operations and, together, the EV and AMM was rated to be of tremendous benefit in maintaining SA and workload during 300 RVR approach and departures with simulated taxi-in and -out. The results also fully support the potential direction that EV with an AMM may provide an “operational credit” for SMGCS wherein an operator, with these requisite flight deck technologies, may be able to conduct lower than 500 RVR operations at airports that may only have a Level 1 LVO/SMGCS airport visual aids in place. Another option may be to enable under 1200 RVR surface operations at airports that do not have any LVO/SMGCS airport visual aids in place.
The FAA has stated that, “taxiing on the airport surface is the most hazardous phase of flight” (Gerold, 2001). Almost a decade later, that statement still rings true, but LVO/SMGCS enhanced visual aids and other controls are significantly improving this situation. Emerging flight deck technologies offer a potential means to create an equivalent level of safety and performance. These flight deck technologies, such as the E-SMGCS -AMM display and EV, could assist in fully realizing the potential of NextGen by offering a more affordable path toward safe and efficient LVO/SMGCS operations through an “equivalent visual” paradigm. \citep{prinzel2013}

\emph{TODO:}
The display concepts tested in this experiment – typical of current and future PF HUD and PNF-AD displays
– showed poor incursion detection functionality. Only one of the runway incursion scenarios was detected
through use of the cockpit displays. Therefore, requirements for display and sensor technology for runway
incursion detection should be developed which span the breadth of the problem, including human perception,
sensor design and detection theory, crew procedures, and crew interface issues. Current flight crews are not
familiar with using head-down displays on short final to check for incursions. The displays are not
Fusion of Synthetic and Enhanced Vision
11 - 22 RTO/HFM-141
necessarily optimized for this role. This role was not intentionally included in the pre-experiment crew
briefing. \citep{baileyym2007}

\emph{TODO:}
Taxi Performance
Advances in HUD technology have extended its use to taxi guidance. During taxiing,
pilots rely heavily on visual navigation aids on the airport surface outside the
cockpit. Helping pilots to taxi during times of low visibility could, therefore, increase
the safety and efficiency of surface movements. However, as pointed out by
M. Wiggins (personal communication, September 4, 2003), the effectiveness of usingHUDsfor
taxi guidance depends on the accuracy and reliability of the route path
programming. It would therefore be necessary to include safeguards to ensure that
pilots do not blindly follow HUDs for example, onto an inactive runway.
A number of studies assess how effectively HUDs provide taxi guidance
(Battiste, Downs, \& McCann, 1996; Jones, Quach, \& Young, 2001). In
low-visibility conditions, poor taxi performance is dangerous, increases fuel costs
to airlines, and affects passenger schedules. McCann, Andre, Begault, Foyle, and
Wenzel (1997) examined taxi performance when pilots were using T–NASA’s
“Scene-Linked” HUDs and 3–D moving maps. The HUDs substantially improved
taxi performance over performance seen with 3–D moving maps. \citep{crawford2006}







































































































\chapter{Järjestelmien käyttöön liittyviä haasteita} \emph{TODO:}

Tässä kappaleessa kerrotaan erilaisista ongelmista, joita järjestelmien käyttöönotossa on otettava huomioon.

\section{Järjestelmien käyttöönotto vanhoissa ja uusissa ohjaamoissa} \emph{TODO:}

One point concerns the number of displayable objects. This
number is confined because of limitations in graphic hardware
performance and memory capacity. Another point is related to
texturing. Present computer hardware which is suited for use
onboard of an aircraft has also no real-time texture capabilities.
The lack of texturing may cause difficulties for the pilot in
estimating the height above ground when the terrain surface is
LOD \citep{mollersachs1994}

Tässä kerrotaan lähinnä teknisistä haasteita, joita voi esiintyä otettaessa järjestelmiä käyttöön vanhoissa sekä uusissa ohjaamoissa. Luonnollisesti uusissa ohjaamoissa voidaan ottaa SVS huomioon jo niitä suunniteltaessa.

The retrofit question concerns whether useful and effective synthetic vision displays
are usable in aircraft that have limited-size display spaces. Two experiments were
conducted to examine the efficacy of these displays and develop field-of-view and
terrain texture recommendations for design. The first experiment examined issues of
field of view and display size using an Asheville, North Carolina, synthetic vision database
and fixed-based simulator. The second experiment was conducted on the
NASA B-757 aircraft at Dallas/FortWorth International Airport and investigated the
efficacy of both head-down and head-up displays and generic and photorealistic terrain
texture. Both experiments confirmed the retrofit capability and that all sizes and
texturing methods were found to be viable candidates for synthetic vision displays.
These results, future directions, and implications for meeting national aeronautic
safety and capacity goals are discussed. \citep{prinzel2004}

Although the safety and economic advantages and payoff to pursuing SVS are
great, there are significant research challenges to be addressed before SVS can be
considered viable as a technological alternative. To provide a better definition of
the concept of operations (CONOPS) of synthetic vision technology for commercial
and business aircraft, a workshop resulting in a CONOPS document was held
at the NASA Langley Research Center (Williams et al., 2001). The focus of this
event was to obtain wide-ranging input from the aviation community on the benefits
and features that synthetic vision might incorporate. The outcome of the workshop
and subsequent activities has been the identification of numerous challenges
and research issues that need to be explored in developing SVS display concepts.
Many of these issues can be classified as human perceptual, such as display size
and field-of-view (FOV) issues.
The issue of display size is driven largely by the need for displays compatible in
size with current aircraft displays (the retrofit issue) and potential next-generation
larger display surfaces (forward fit issue). Because current aircraft have either
electromechanical instruments (e.g., 737–200) or small “glass” displays (e.g.,
757–200), there are concerns about the efficacy of these cockpits to support SVS
because of the physically smaller instrument spaces. One option to address the retrofit
issue would be to present SVS on a head-up display (HUD), and research
questions turn to how best to display synthetic terrain on a HUD that has limited
graphical capabilities. Another option is to simply remove the traditional instruments
and replace them with synthetic vision displays, and research issues then
turn to whether the space constraints will allow SVS presentations to be usable by the flight crew. Because these displays have a small unity geometric FOV, the scale
factor might need to be increased (i.e., minified) to allow more of the visual scene
to be presented to make the SVS display effective (e.g., Roscoe, 1948). The
wide-angle lens effect of increasing FOV, however, interacts with display size and
can lead to perceptual distortions as the minification factor (MF) is increased (e.g.,
virtual space effect; McGreevy \& Ellis, 1986). \citep{prinzel2004}

\section{Ohjaajan suorituskyky ja kognitiiviset haasteet}

\emph{TODO:} PROTO 

\chapter{SV- ja EV-järjestelmien tulevaisuus}

\section{Tulevaisuuden sovelluksia}

Tässä pohditaan, minkälainen tulevaisuus SV- ja EV-järjestelmiä mahdollisesti odottaa kaupallisen ilmailun näkökulmasta.

\emph{TODO:}
Based on our experimental data, it seems that SVIS displays such as our Condition
2 may find application in situations where the precision of aircraft navigation is of
utmost importance. Such situations could include commercial airline approaches
into terrain-challenged airports, where curved paths may be more effective in
maintaining maximum terrain separation than would be possible with conventional
approaches. Another application may be in situations where closely spaced
parallel approaches are conducted to parallel runways with minimal lateral separation. \citep{schnellym2004}

The primary key to
providing SVIS for application in the real world seems to be a refined guidance
concept with a carefully tuned flight path predictor and a guidance cue that is tuned
to the predictive nature of the flight path predictor. The depiction of the terrain is
probably a secondary key point that adds to the usefulness of the SVIS. \citep{schnellym2004}

\emph{TODO:}
EMERGING ISSUES AND APPLICATIONS
There is an abundance of new avionics available now and these are provided by
a multitude of manufacturers. For example, in 2002 Boeing conducted demonstrator
flights in a specially modified 737–900. The advanced avionics to be
evaluated by representatives from a variety of airlines included synthetic vision
system (SVS) displays; a head-up guidance system married to two infrared enhanced
vision systems (EVS); highway-in-the-sky navigational cues; a virtual-
traffic-cone surface guidance system; global positioning satellite (GPS)
landing system; and a software upgrade to the enhanced ground proximity warning
system (EGPWS). Other prevention technologies include NASA’s Runway
Incursion Prevention System, which alerts pilots on approach to other aircraft
that pose a threat. In the following sections, we consider the implications of
these types of emerging technologies for the use of HUDs.
Pathway HUDs
Pathway, tunnel, or highway-in-the-sky 3–D format flight path displays provide
a prediction and preview of a flight path. Although these displays have been in-vestigated for a few decades, combining the pathway display and HUD is a relatively
new concept.
Pathway HUDs have been investigated by Ververs and Wickens (1998) and
Fadden, Ververs, and Wickens (2000). The three elements that make up the new
HUD are a preview tunnel of where the aircraft will be in the future, a predictor
symbol, and a 3–D perspective. Preliminary tests with pilots showed positive performance
results, in the form of more precise tracking with lateral and vertical error
limited to approximately 10 ft. However, when an unexpected event arose for
pilots who were truly naive to the study (in this case a runway incursion), the event
was detected more slowly in the HUD condition than in the HDD condition. It
should be noted that this finding was not statistically significant, possibly due to
the insufficient power of the study. Nevertheless, the results are consistent with the
cognitive tunneling effects reported previously, without pathway displays. \citep{crawford2006}



\section{NextGen ja EVO-konseptit}

\emph{TODO:}
Tässä osiossa kerrotaan NextGen-ilmailukonseptista sekä EVO-konseptista sekä näönparannusjärjestelmän vaatimuksista, joita näihin konsepteihin on visioitu.

\emph{TODO:}
Tämä saattaa olla hyvä laittaa edellisen osion alle (tai sitten ei), katsottava muotoiluvaiheessa.

\emph{TODO:}
NASA is striving to develop the technologies and knowledge to enable EVO and to extend EVO towards a “Better-Than-Visual” operational concept. This operational concept envisions an "equivalent visual" paradigm where an electronic means provides sufficient visual references of the external world and other required flight references on flight deck displays that enable Visual Flight Rules (VFR)-like operational tempos while maintaining and improving safety of VFR while using VFR-like procedures in all-weather conditions.\citep{prinzel2013}

\emph{TODO:}
Taken together, the conclusions that can be drawn from these experiments are that
synthetic vision can be implemented on retrofit sizes and, therefore, can successfully
be introduced into the current aircraft fleet. To be effective, synthetic vision
presented on small display sizes would have to be minified and our results indicate
that the MF should not exceed 4.5 for optimal performance, although more research
is needed to confirm such a conclusion.
Because no performance differences were found between photo-realistic and
generic terrain texture methods, the generic terrain presentation might represent an
effective and lower cost option for synthetic vision displays, although photo-realistic
terrain does have properties that can increase the margin for safety and operations.
Several pilots did comment that photo-realistic texture would be helpful for
SA during climb, en route, and descent phases of flight. A recent flight test in the
terrain-challenged area of Eagle-Vail, CO, however, found no performance or SA
penalties for the generic texture concept, although pilots reported an overall preference
for the photo-realistic presentation (Bailey et al., 2002; Prinzel et al., 2002).
The NASA Aviation Safety Program is currently evaluating a synthetic vision concept
that combines generic and photo-realistic terrain texture to take advantage of
the benefits both methods offer for SA. \citep{prinzel2004}

\emph{TODO:}
The problem of reduced visibility challenges aviation goals to reduce the accident
rate and improve operational capacity (Federal Aviation Administration, 2001;
NASA, 2001). The approach of synthetic vision is to solve the problem through the
presentation of how the outside world would look to the pilot if vision were not restricted.
TAWSare steps in the right direction and they have significantly improved
safety, but the solution treats the symptoms and not the cause (Moroze \& Snow,
1999). Synthetic vision instead provides for proactive prevention of visibility-induced
accidents while also increasing the capability to make approaches in
weather conditions and airports not currently legal for low-visibility operations.
Although our research did not specifically address these aviation safety and operational
benefits, subsequent studies (e.g., Prinzel et al., 2002) have substantiated the
performance and SA enhancements of synthetic vision even while making complex,
circling approaches under conditions that are beyond current cockpit technology
capabilities. Furthermore, the concept described here represents only the
database and display concepts and not the total SVS, which will include synthetic
vision navigation displays; runway incursion prevention technology; database integrity
monitoring equipment; enhanced vision sensors; taxi navigation displays;
and advanced communication, navigation, and surveillance technologies
(McCann et al., 1998; Timmerman, 2001; Uijt de Haag, Young, Sayre, Campbell,
\&Vadlamani, 2002;Williams et al., 2001; Young \& Jones, 2001). These technologies
represent a comprehensive solution that will be evaluated in near-term NASA
simulation and flight research. Together, synthetic vision may considerably help
meet national aeronautic goals to “reduce the fatal accident rate by a factor of 5”
and to “double the capacity of the aviation system,” both with 10 years (NASA,
2001, p. 2). \citep{prinzel2004}


\chapter{Yhteenveto} \emph{TODO:}

Synthetic vision has the potential to provide significant safety and economic benefits,
particularly if the system is effective as both a retrofit and forward-fit solution
to visibility-restricted problems. Previous research has shown the efficacy of synthetic
vision on large-size displays and, therefore, synthetic vision is expected to
be capable of effective presentation as glass displays become larger with each generation
of aircraft. Because the majority of the current commercial aircraft fleet has
electromechanical instruments or limited glass real estate, however, any significant
benefits would require answering the retrofit question of whether effective
presentation of synthetic vision can also be made in current aircraft cockpits.
Display Size
The retrofit question concerns our hypothesis that the HUD and smaller SVS display
sizes would provide adequate information to enable the pilot to make safe and
precise approaches. The results of the experiments confirmed the hypothesis and
suggest that SVS is viable as a retrofit candidate. Experiment 1 showed no differences
in path performance between display sizes or FOV, and Experiment 2
showed differences only for the HUD concept for lateral path performance. \citep{prinzel2004}

Analysis results indicate that the SVIS is superior to the conventional displays
for the majority of the measures that were obtained in the experiments.
Based on the results of the experiments, we feel that the SVIS is a display format
that improves the performance of pilots in many ways. \citep{schnellym2004}

In summary, research suggests that there are a number of advantages in using
HUDs. These include increases in flight path tracking accuracy, except during
cruise flight; benefits for event detection, except in the approach and landing
phase and for unexpected events; lower visibility takeoff and landing; more accurate
approach and landing; the elimination of head-down time; a reduction in
the time taken to refocus between instruments and the external scene; and the
potential to use overlaid symbology for the external scene when it is not visible,
hence enhancing situation awareness. The major disadvantages of HUDs are difficulties
in switching attention between the internal and external scene and difficulties
in detecting unexpected events. Despite this, there is currently no evidence
to suggest that the use of HUDs is associated with an increased risk of
accidents.
Withthe use ofHUDsincreasing throughoutcommercialaviation,moreresearch
is needed to identify ways to address the issues just noted. From a practical perspective,
one of the most pressing issues concerns training.Wedo not yet know whether
it is possible to train pilots to overcome the effects of cognitive tunneling, or how
muchtraining would be needed if itwaspossible to do so.Oneoption is to train pilots
to scan more effectively by teaching them to take their attention away from theHUD
and into the far domain (Wickens, Helleberg, Goh, Xu,\&Horrey, 2001). We could
16 CRAWFORD AND NEAL
Downloaded by [Jyvaskylan Yliopisto] at 07:02 22 December 2013
not find any studies in the public domain that have addressed this issue. Other issues
that need to be examined include the use of HUDs in multicrew operations, and the
effects of fatigue and experience on performance when using an HUD. Although
HUDsoffer significant benefits to airlines in both productivity and safety, an extensive
program of research is needed to ensure that these gains are realized. \citep{crawford2006}

Results of this study suggest that automation bias is a significant factor in pilot
interaction with automated aids, and that pilots are not utilizing all available
information when performing tasks and making decisions in conjunction with
automation. Pilots exhibited the same overall rate of automation-related errors as
the student population in the Skitka et al. (1996) study, demonstrating that expertise
does not insulate individuals from automation bias. In fact, experience and expertise,
which might be predicted to make pilots more vigilant and less susceptible to
automation bias, were related to a greater tendency to use only automated cues. One
possible explanation for this may be that, because automated systems tend to be
highly reliable, more experience with them reinforces the notion that other cues are
merely redundant and unnecessary. Additionally, the higher-experienced pilots in
this study tended to be those more senior within their airline, and were currently
flying as captains. They may be used to delegating the cross-checking role to their
first officers rather than doing it themselves.
Although this study does not provide evidence that externally imposed accountability
affects the decision-making behavior of professional pilots, it does demonstrate
that the internalization of "accountability" for performance and strategies in
the use of automated systems impacts automation bias. Perceived accountability
was positively correlated with increased verification of automated functioning and
fewer omission errors. In other words, pilots who reported a higher internalized
sense of accountability for their interactions with automation verified correct
automation functioning more often and committed fewer errors than other pilots.
These results suggest that the sense that one is accountable for one's interaction
with automation encourages vigilance, proactive strategies, and the use of all
information in interactions with automated systems. The fact that the perception of accountability was not correspondent with our external manipulation indicates the
need to establish the degree to which accountability is a variable that can be
significantly influenced in pilots and other professional decision makers, who are
already functioning at a high level of personal responsibility for their conduct.
Alternatively, the lack of experimentally determined accountability effects could
be the artifact of a small sample, or the experimental manipulation could have been
confounded by a sense of accountability induced by the experience of participating
in a NASA study. The perception of accountability might also be part of some innate
cognitive style or personality construct, a hypothesis that will be addressed in a
future study.
Several aspects of the experimental tasks seemed to affect the tendency of
participants to verify the functioning of the automation; these included task importance,
predictability, and feedback. Descriptive data suggested that pilots were more
likely to catch automation events that involved altitude and heading than one that
involved a frequency error, corresponding with the typical ordering of flight
priorities as (a) aviate, (b) navigate, (c) communicate. The fact that the tracking
task automation failure so readily captured pilot attention can be explained in part
by the fact that the task display offered trend information (e.g., pilots could see
when the cursor was starting to drift out of the target area), as well as immediate
salient feedback on errors. In many real-world automation-aided tasks, pilots do
not receive either predictive information or immediate feedback on accuracy and
errors. Incorporating these features in the design of new automated systems may
be a way to facilitate vigilant use and aid in the detection of automation failures.
The finding that all pilots responded to the false engine fire event by shutting
down the indicated engine was unanticipated. Even more surprising was the fact
that their actions were contrary to responses on the debriefing questionnaires
indicating that some combination of cues would be necessary to diagnose "definitely
a fire," and that it would be safer, in the presence of only an EICAS message,
to retard the throttle of the indicated engine rather than shutting it down. Apparently,
these pilots were acting contrary to their self-described strategies. Part of the
explanation for their actions may rest in the "false memory7' data.
Pilots were definitely aware of the cues typically associated with an engine fire
and, in this time-pressured and cognitively demanding situation, were evidently
remembering a pattern of cues that was consistent with what should have been
present during the event. Real correlations among the cues--during an engine fire,
all of them should have been present--contributed to the illusion of their presence.
In similar situations during their careers (i.e., engine fire situations calling for the
shutdown of an engine), the pilots in this study would have been wrong if they had
not remembered the presence of several cues3 The more quickly they acted, the more "sure" they had to be that their actions were correct, and the more phantom
cues they remembered as having prompted their actions. Their judgments and their
memories were biased in the direction of confirming their expectations, and, as
Hamilton (1981) aptly remarked concerning correlated cues and expectations,
"[they] wouldn't have seen it if [they] hadn't believed it" (p. 137).
It is possible that this false memory phenomenon may be an additional mechanism
by which automation bias is bolstered. Pilots may not ever be aware that they
are using automated information as a short cut, either because the automated cues
are in fact consistent with other information (and no error results), or because errors
are not traced back to a failure to cross-check automated cues. In the false engine
fire event, for example, most pilots believed they had acted based on several cues,
and thus would not be prompted to change their strategies in the use of automated
information. Consistent with this analysis, pilots tended to falsely remember cues
that fit into their expectations but were not physically handled during the engine
shutdown. For example, they were less likely to remember the presence of an
illuminated fire handle (n = 2), which they had to manually press during engine
shutdown (and would have been forced to observe directly) than a master warning
light (n = 8), which was not directly manipulated as part of the shutdown procedure.
Explicit manipulation of a control made it more salient, and less likely to be
incorrectly remembered. \citep{mosier1998}

CONCLUSIONS
In conclusion, this study has documented the existence of attentional tunneling as a
legitimate concern for the SV display when coupled with the HITS. Such concern certainly does not fully compromise the very real benefits of the HITS in supporting
low-workload trajectory guidance, and particularly of the substantial benefits of the
SV terrain representation for supporting terrain awareness. It does, however, point to
the importance of attentional training for pilots who use this new technology. \citep{wickens2009}

%Lähdeluettelo

\begin{thebibliography}{}

% Hakasulkeisiin tulee kirjoittajien sukunimet (siinä muodossa kuin
% ne halutaan lähdeviittaukseen) ja julkaisun vuosiluku suluissa.
% Huom: Älä laita välilyöntiä ennen vuosiluvun alkusulkua.

% Normaali viittaus ym.-sanalla, ensimmäisessä viittauksessa kaikki nimet:

\bibitem[Bailey ym. (2007)Bailey, Randall E. , Kramer, Lynda J. \& Prinzel, Lawrence, III]{baileyym2007}
Bailey, Randall E. , Kramer, Lynda J. \& Prinzel, Lawrence, III 2007.
\textit{Fusion of Synthetic and Enhanced Vision for All-Weather Commercial Aviation Operations}.
NASA Technical Reports Server (NTRS), huhtikuu 2007.

\bibitem[Beier \& Gemperlein(1994)Beier \& Gemperlein]{beiergemperlein2004}
Beier, Kurt \& Gemperlein, Hans 2004.
\textit{Simulation of Infrared Detection Range at Fog Conditions for Enhanced Vision Systems in Civil Aviation}.
Aerospace Science and Technology, 8, s.~63--71.

\bibitem[Bennet \& Flach(1994)Bennet \& Flach]{bennetflach1994}
Bennet, K.B. \& Flach, J.M. 1994.
\textit{When automation fails …}.
Human performance in automated systems: Current research and trends, s.~229--234.

\bibitem[Boeing(1996)Boeing]{boeing1996}
Boeing 1994.
\textit{Statistical summary of commercial jet aircraft accidents, worldwide operations,
1959–1995}.
Seattle, WA: Airplane Safety Engineering, Boeing Commercial Airplane Group.

\bibitem[Crawford \& Neal(2006)Crawford, Jennifer \& Neal, Andrew]{crawford2006}
Crawford, Jennifer \& Neal, Andrew 2006.
\textit{A Review of the Perceptual and Cognitive Issues Associated With the Use of Head-Up Displays in Commercial Aviation}
The International Journal of Aviation Psychology, 16:1, s.~1--19.

\bibitem[Etherington ym. (2000)Etherington, T.J., Vogl, T.L., Lapis, M.B., \& Razo, J.G.]{etherington2000}
Etherington, T.J., Vogl, T.L., Lapis, M.B., \& Razo, J.G. 2000.
\textit{Synthetic vision information system}
Proceedings of the 19th Digital Avionics Systems Conference, s.~2.A.4--2.A.8.

\bibitem[Hooey \& Foyle(2007)Hooey, B.L. \& Foyle, D.C.]{hooey2007}
Hooey, B.L. \& Foyle, D.C. 2007.
\textit{Aviation Safety Studies: Taxi Navigation Errors and Synthetic Vision Systems Operations}
Human Performance Modeling in Aviation.

\bibitem[Mosier ym. (1998)Mosier, Kathleen L. , Skitka, Linda J. , Heers, Susan, \& Burdick, Mark]{mosier1998}
Mosier, Kathleen L. , Skitka, Linda J. , Heers, Susan, \& Burdick, Mark 1998.
\textit{Automation Bias: Decision Making and Performance in High-Tech Cockpits}
The International Journal of Aviation Psychology, 8:1, s.~47--63.

\bibitem[Möller \& Sachs (1994)Möller \& Sachs]{mollersachs1994}
Möller, H. \& Sachs, G. 1994.
\textit{Synthetic Vision for Enhancing Poor Visibility Flight Operations}.
IEEE AES Systems Magazine, maaliskuu 1994, s.~27--33.

\bibitem[Nordwall(1993)]{nordwall1993}
Nordwall, B.D. 1993.
\textit{HUD with IR System extends Pilot Vision}.
Aviation Week \& Space Technology, helmikuu 1993, s.~62--63.

\bibitem[Prinzel ym. (2013)Prinzel, Lawrence J. III, Arthur, Jarvis J. , Kramer, Lynda J. ,  Norman, Robert M. , Bailey, Randall E. , Jones, Denise R. ,  Karwac, Jerry R. Jr. , Shelton, Kevin J. \& Ellis, Kyle K. E.]{prinzel2013}
Prinzel, Lawrence J. III, Arthur, Jarvis J. , Kramer, Lynda J. ,  Norman, Robert M. , Bailey, Randall E. , Jones, Denise R. ,  Karwac, Jerry R. Jr. , Shelton, Kevin J. \& Ellis, Kyle K. E. 2013.
\textit{Flight-Deck Technologies to Enable NextGen Low Visibility Surface Operations}.
NASA Technical Reports Server (NTRS), toukokuu 2013.

\bibitem[Prinzel ym. (2004)Prinzel, Lawrence J. III, Comstock, J. Raymond Jr. , Glaab, Louis J. , Kramer, Lynda J. , Arthur, Jarvis J. & Barry, John S.]{prinzel2004}
Prinzel, Lawrence J. III, Comstock, J. Raymond Jr. , Glaab, Louis J. , Kramer, Lynda J. , Arthur, Jarvis J. \& Barry, John S. 2004.
\textit{The Efficacy of Head-Down and Head-Up Synthetic Vision Display Concepts for Retro and Forward-Fit of Commercial Aircraft}
The International Journal of Aviation Psychology, 14:1, s.~53--77.

\bibitem[Schnell ym. (2004)Schnell, Thomas,  Kwon, Yongjin, Merchant, Sohel & Etherington, Timothy]{schnellym2004}
Schnell, Thomas,  Kwon, Yongjin, Merchant, Sohel \& Etherington, Timothy 2004.
\textit{Improved Flight Technical Performance in Flight Decks Equipped With Synthetic Vision Information System Displays}
The International Journal of Aviation Psychology, 14:1, s.~79--102.

\bibitem[Ververs \& Wickens (2004)Ververs, Patricia May \& Wickens, Christopher D.]{ververs1998}
Ververs, Patricia May \& Wickens, Christopher D. 1998.
\textit{Head-Up Displays: Effect of Clutter, Display Intensity, and Display Location on Pilot Performance}
The International Journal of Aviation Psychology, 8:4, s.~377--403.

\bibitem[Wickens \& Alexander(2009)Wickens, Christopher D. \& Alexander, Amy L.]{wickens2009}
Wickens, Christopher D. \& Alexander, Amy L. 2009.
\textit{Attentional Tunneling and Task Management in Synthetic Vision Displays}
The International Journal of Aviation Psychology, 19:2, s.~182--199.

\bibitem[Wickens \& Andre(1990)Wickens, C.D. \& Andre, A.D.]{wickensandre1990}
Wickens, C.D. \& Andre, A.D. 1990.
\textit{Proximity compatibility principle and information display: Effects
of color, space, and objectness on information integration}
Human Factors, 32, s.~61--77.

\bibitem[Williams ym. (2001)Williams, D. , Waller, M. , Koelling, J. , Burdette, D. , Doyle, T. , Capron,W. , ym.]{williamsym2001}
Williams, D. , Waller, M. , Koelling, J. , Burdette, D. , Doyle, T. , Capron,W. , ym. 2001.
\textit{Concept of operations for commercial and business aircraft synthetic vision systems}
NASA Tech. Memo. No. TM-2001-211058.

\end{thebibliography}

\end{document}
